%% 
%% Copyright 2019-2024 Elsevier Ltd
%% 
%% Version 2.4
%% 
%% This file is part of the 'CAS Bundle'.
%% --------------------------------------
%% 
%% It may be distributed under the conditions of the LaTeX Project Public
%% License, either version 1.2 of this license or (at your option) any
%% later version.  The latest version of this license is in
%%    http://www.latex-project.org/lppl.txt
%% and version 1.2 or later is part of all distributions of LaTeX
%% version 1999/12/01 or later.
%% 
%% The list of all files belonging to the 'CAS Bundle' is
%% given in the file `manifest.txt'.
%% 
%% Template article for cas-dc documentclass for 
%% double column output.

%\documentclass[a4paper,fleqn,longmktitle]{cas-dc}
\documentclass[a4paper,fleqn]{cas-sc}

%\usepackage[authoryear,longnamesfirst]{natbib}
%\usepackage[authoryear]{natbib}
\usepackage[numbers]{natbib}
\usepackage{comment} % Added for comment environment
\usepackage{etoolbox} % Added for \csdef command
\usepackage{xspace} % Added to support \xspace used in \tsc
\usepackage{subcaption} % Added for subfigures


%%%Author definitions
\def\tsc#1{\csdef{#1}{\textsc{\lowercase{#1}}\xspace}}
\tsc{WGM}
\tsc{QE}
\tsc{EP}
\tsc{PMS}
\tsc{BEC}
\tsc{DE}

% Define tblwidth if not already defined
\providecommand{\tblwidth}{\textwidth}
%%%

\begin{document}
\let\WriteBookmarks\relax
\def\floatpagepagefraction{1}
\def\textpagefraction{.001}
\shorttitle{Sparse defect localization  -  a periodic Empirical Operator Framework for Inverse Scattering and Structural Health Monitoring}
\shortauthors{R. G. Wueest, H. R Thomsen, J. Aichele}

\title [mode = title]{Sparse defect localization  -  a periodic Empirical Operator Framework for Inverse Scattering and Structural Health Monitoring}
%\title [mode = title]{Fully Vectorized Linear Operator Formulation for Inverse Scattering Problems}                      


\author[1]{R. G. Wueest}[type=editor,
                        role=Researcher]
\cormark[1]
\fnmark[1]
\ead{richard.wuest@protonmail.com}

\credit{Conceptualization, Software and Evaluation, Writing – original draft}

%\address[1]{, Street 129, 1043 NX Amsterdam, The Netherlands}
\affiliation[1]{organization={Department of Physics, ETH Zürich},
%               citysep={}, % Uncomment if no comma needed between city and postcode
                postcode={8092}, 
                state={Zürich},
                country={Switzerland}}

%\author[2]{J. Mueller}[role=Advisor]
%\credit{Writing - review \\\& editing}


\affiliation[2]{organization={Department of Earth Sciences, ETH Zürich},
                postcode={8092}, 
                state={Zürich},
           country={Switzerland}}

% Additional advisors from thesis (placeholders where details absent)
\author[2]{J. Aichele}[role=Advisor]
\credit{Methodology, Investigation, Writing - review \& editing}


\author[2]{H. R. Thomsen}[role=Advisor]
\fnmark[2]
\ead{thomsenh@ethz.ch}
\credit{Supervision, Methodology, Investigation , Resources, Writing - review \& editing}


\begin{abstract}
We present an empirically calibrated, fully vectorized linear-operator framework for inverse scattering, motivated by challenges in non-destructive testing (NDT) and structural health monitoring (SHM) where reliable forward models are difficult to obtain and maintain. The method enables defect localization without requiring an explicit physical model of wave propagation. Source–receiver impulse-response operators are constructed through a one-time calibration using bonded piezoelectric transducers for excitation and dense Laser Doppler Vibrometer (LDV) measurements. Enforcing discrete periodic time behavior represents propagation as circular convolution, yielding circulant operators diagonalizable via the discrete Fourier transform (DFT) and allowing defect localization to be posed as a Tikhonov-regularized linear inverse problem using only sparse monitoring data.

Experimental validation on a bidirectional carbon-fiber-reinforced polymer (CFRP) plate demonstrates localization of a 1 cm³ adhered defect using measurements acquired solely at transducer locations after calibration. Computational cost is concentrated in the calibration stage, while subsequent evaluations reduce to rapid matrix–vector operations. Additional simulations complement the experimental results and probe performance across defect scenarios and sensing configurations, including proof-of-concept localization in a single-transducer (pulse-echo) setting.


Relying only on linear time-invariant system assumptions, the framework provides a data-driven alternative for NDT and SHM in situations where maintaining accurate digital twins is impractical, and may extend to other wave-based sensing applications.
\end{abstract}

\begin{graphicalabstract}

\end{graphicalabstract}

\begin{comment}
\begin{graphicalabstract}
\includegraphics{thumbnails/cas_email.jpeg}
\end{graphicalabstract}
\end{comment}

\begin{highlights}
\item Presents a fully vectorized linear operator framework for inverse scattering in NDT.
\item Relies solely on the principles of linear time-invariant (LTI) systems, avoiding the need for theoretical assumptions about wave propagation.
\item Enforces periodic discrete time, enabling FFT-diagonalizable circular convolution operators which are solvable using the Tikhonov regularized inverse.
\item Localizes defects in both simulated aluminum plates and real-world CFRP plates.
\item Post-calibration, only sparse sensor measurements are needed for rapid health monitoring.
\end{highlights}

\begin{keywords}
Inverse scattering \sep Ultrasonic guided waves \sep Laser Doppler vibrometry (LDV) \sep Empirical impulse response \sep Circular convolution \sep Tikhonov regularization \sep Structural health monitoring
\end{keywords}


\maketitle

\section{Introduction}\label{Introduction}


Structural Health Monitoring (SHM) of engineered components seeks to localize and characterize defects before they compromise structural integrity. The relevance of SHM is closely linked to the cost of failure, which is particularly critical in aviation, and has motivated extensive research on monitoring approaches for relatively simple, flat structures \cite{Aviation_Book}. Among the available techniques, piezoelectric transducers are attractive from a commercial perspective because of their low cost and ease of integration. Nevertheless, significant difficulties arise when these concepts are transferred to carbon-fiber reinforced polymers (CFRPs) \cite{Aircraft_NDT}, and a range of dedicated strategies for CFRP inspection have therefore been investigated \cite{Current_Methods,Composite_pipe_ndt}. Interpreting guided-wave measurements in such materials is inherently complex: overlapping modes often demand dedicated processing, such as ridge-based time–frequency analysis, beamforming, or migration to extract meaningful arrivals \cite{Raghavan2007}, while benign geometric features, multiple reflections, and mode conversions at joints and thickness transitions generate coherent clutter that can obscure damage signatures \cite{Cawley2002,Olisa2021}. These effects are amplified by the pronounced anisotropy, laminate heterogeneity, and viscoelastic damping of CFRPs, which lead to direction- and frequency-dependent dispersion and attenuation \cite{Calomfirescu2008,Ostachowicz2024}. Curvature, stiffeners, and adhesive interfaces introduce additional localized scattering, and in large curved assets such as rocket fuselages or wind-turbine blades, spatially varying thickness and fiber orientation challenge travel-time–based concepts \cite{Yu2025,Chia2023,Jankauskas2011,Mustapha2024,Park2017}.

Sparse monitoring of Lamb waves using a limited number of piezoelectric sensors \cite{Lamb_wave_transducer} would therefore be highly attractive. However, such approaches generally rely on detailed numerical forward models. Modern numerical frameworks, including finite- and spectral-element solvers combined with higher-order plate formulations have proven capable of reproducing dispersion, mode conversion, and multi-path scattering in anisotropic laminates with high fidelity while remaining computationally efficient at ultrasonic frequencies \cite{Calomfirescu2008,Kudela2024}. Nevertheless, such models are difficult to maintain for composite structures \cite{Modelling_Challenges,OG_Lamb_wave}. CFRP components, in particular, exhibit variability arising from lay-up and manufacturing processes that undermines the reliability of predictive digital twins \cite{Manufacturing_Variability_Problems,CFRP_Defects}. In response to these limitations, data-driven strategies based on Laser Doppler Vibrometry (LDV) have gained increasing attention as a means to capture rich wavefield information directly from the structure and to reduce dependence on uncertain a priori models \cite{LDV_SHM_CFRP, Thomsen2024}.

This paper validates a novel method based on empirical linear operators\cite{Matrix_use_in_ndt}\cite{Touchscreen} and a fundamentally periodic time formulation. The result is a matrix based description of the system and it's inversion to solve for defects \cite{inversion_method_1}. We excite the structure with bonded piezoelectric transducers and use an LDV to perform a dense, one-time calibration LDV scan that constructs source–receiver impulse-response operators. Wave propagation is treated on a discrete cyclic time axis so that it acts as circular convolution, diagonalized by the discrete Fourier transform (DFT). 
By designing the excitation to be strictly periodic and ensuring no other sources act on the object, the cyclic time embedding is valid by construction. We impose time periodic behavior in the setup, and thus the frequency treatment becomes equivalent to the time domain treatment.
Within a single-scattering formulation, defect localization is solved via Tikhonov-regularized inverse\cite{Tikhonov}\cite{Tikhonov_matrix} using only sparse monitoring data at the transducer locations. We validate the approach experimentally on an anisotropic CFRP plate.

\section{Mathematical framework}\label{Mathematical framework}
Before presenting the experimental results, we first introduce the mathematical framework underlying the proposed defect-detection approach.

\subsection{System assumptions}\label{System assumptions}
We assume: (i) linearity; (ii) time invariance; (iii) reciprocity; (iv) practical flatness (LDV access); and (v) the ability to control the excitation of the material relative to noise.

Assumptions (i), (ii), and (iii) are naturally satisfied for mechanical waves under standard conditions. Assumption (iv) refers to geometries where the object thickness is small relative to its lateral dimensions, typically valid for plate-like components such as aircraft fuselage panels and shell structures. Assumption (v) simply requires a controlled excitation capability, which is standard in active NDT/SHM configurations using piezoelectric or electromagnetic transducers.

\subsection{Terminology and Notation}
We work with discrete-time vectors and matrices throughout. The spatial domain is discretized into a set of points indexed by integers without implying any particular geometric arrangement. For instance, a 100×100 scan grid of a square plate yields 10,000 points labeled 1 through 10,000, with the specific mapping arbitrary, but fixed.

When a transducer emits a signal at location $i$, represented by time-domain vector $\vec{s}_i$, the observed signal $\vec{m}_j$ at location $j$ is given by the convolution with the impulse response $\vec{g}_{ij}$ between those locations: $\vec{m}_j = \vec{g}_{ij} \ast \vec{s}_i$. This formulation separates spatial indices (subscripts) from the time-domain representation (vectors).

For computational implementation, these vectors correspond to discrete-time samples over a fixed acquisition window, with all operations respecting the imposed periodicity described in subsequent sections.


\subsection{Periodic time and discrete representation}\label{Periodic time and discrete representation}

Repeating the excitations enforces periodicity so that the time axis can be treated on a discrete cyclic domain. Any impulse response between locations $i$ and $j$ is then a circular-convolution operator $G_{ij}$, represented by a circulant (circular Toeplitz) matrix fully determined by its first column (the sampled impulse-response vector $\vec g_{ij}$). Explicitly, for $\vec g_{ij} = (g_0, g_1, \ldots, g_{n-1})^T$,
\begin{equation}
G_{ij} = \begin{pmatrix}
g_0 & g_{n-1} & \cdots & g_1 \\
g_1 & g_0 & \cdots & g_2 \\
\vdots & \vdots & \ddots & \vdots \\
g_{n-1} & g_{n-2} & \cdots & g_0
\end{pmatrix}
\end{equation}
so that $G_{ij}\vec v = \vec g_{ij} * \vec v$ (circular convolution). All such operators are diagonalized by the discrete Fourier transform (DFT), enabling component-wise multiplication in the frequency domain.



\subsection{Empirical impulse responses}\label{Empirical impulse responses}
For a transducer $s$ and measurement point $i$, the response is modeled as a circular convolution:
\begin{equation}\label{eq:empirical_conv}
\vec m_{i} = G_{is} \vec s_s = \vec g_{is}\ast\vec s_s
\end{equation}
where $G_{is}$ is circulant (first column $\vec g_{is}$). 

Consider a transducer at location $s$. We periodically emit a test pulse, and the LDV measures the response during a Period over the entire object. The empirical impulse response $\vec{g}_{is}$ relates the measured responses at $s$ and $i$ through equation \ref{eq:empirical_conv}, not the input test pulse itself, hence "empirical" impulse response.

Using commutativity of convolution, rewrite equation \ref{eq:empirical_conv} as
\begin{equation}
    \vec{m}_i = \vec s_s\ast\vec g_{is} = S_s\vec{g}_{is}.
\end{equation}
We estimate $\vec{g}_{is}$ by Tikhonov-regularized\cite{tikhonovintro}\cite{Tikhonov} least squares from multiple (5 in this paper) noisy pulse acquisitions between each sensor and grid index.

Let $(\vec s^p_{s}, \vec m^p_{i s})_{p=1}^P$ denote $P$ repeated acquisitions. The Tikhonov-regularized least-squares problem is
\begin{equation}\label{eq:objective_imp_resp}
\text{argmin}_{\vec{g}_{is}} \;\sum_{p=1}^P \|{\vec m_i^p}-S_s^p\vec{g}_{is} \|_2^2 + \alpha \|  {\vec{g}_{is}} \|_2^2.
\end{equation}
The solution is computed efficiently component-wise in the frequency domain. For each frequency bin $f$ (hats denote DFT components), we rewrite equation \ref{eq:objective_imp_resp}
 to \ref{eq:objective_freq} and solve:
 \begin{equation}\label{eq:objective_freq}
\min_{\hat{g}_{is}(f)} \sum_{p=1}^P \left| \hat{m}_{is}^p(f) - \hat{s}_{s}^p(f) \hat{g}_{is}(f) \right|^2 + \alpha |\hat{g}_{is}(f)|^2
\end{equation}
with regularization parameter $\alpha > 0$. The closed-form solution is: 
\begin{equation}
\hat{g}^{\mathrm{reg}}_{is}(f) = \frac{\sum_p \overline{\hat{s}_{s}^p(f)}\, \hat{m}_{is}^p(f)}{\sum_p |\hat{s}_{s}^p(f)|^2 + \alpha}.
\end{equation}
The resulting stabilized impulse response does not require an explicit noise model. We use a small $\alpha$ (typically $\approx 10^{-7}$) to avoid singularities; within this range, the final solution is insensitive to the exact value.
%the source signal Ss used, comprises the whole time domain signal recorded at the source position. I.e., we do not differentiate between injected energy by the piezo and measured reverberations.  
 

\subsection{Defect model}\label{Defect model}
Since defects also satisfy the assumptions in Section~\ref{System assumptions}, their effects can be modeled using circular convolutions. Within the single-scattering (Born-type) approximation, the scattering process is illustrated in Fig.~\ref{fig:Diag}. We assume a \emph{delta-like} temporal reflection kernel, so each defect is represented by a scalar amplitude $d_k$ at its spatial location. Algebraically, this corresponds to selecting only the first canonical temporal basis vector $\vec d_ k := d_k\vec e_0 = (d_k, 0, 0, \dots)^T$, and physically, it corresponds to a defect position re-emitting the incoming wave immediately with an amplitude $d_k$.  See section \ref{Generalization to arbitrary defect bases} for a generalization of this approach.

\begin{figure}
  \centering
  \includegraphics[width=0.5\linewidth]{Figures/Misc/diagram.png}
  \caption{Visualization on how a source at location $j$ propagates to a defect at location $k$, which then re-emits the wave. Afterwards, it propagates to a receiver location $i$. As an equation, the image visualizes $\vec{m}_i = \vec{g}_{kj}\ast\vec{d}_k\ast\vec{g}_{ik}\ast\vec{s}_j$.}
  \label{fig:Diag}
\end{figure}


% The model in general can be expanded upon to richer reflection models, with any basis of vectors as potential solutions, and has been implemented to with this capability, but no useful bases have been found (That at least justify the higher computational cost), and as such, all results shown in this paper use the simple delta basis.  The generalization to richer local bases is presented in Section~\ref{Generalization to arbitrary defect bases}.





\subsection{Operator assembly}\label{Operator assembly}
For an emitter at position $j$ and a measurement at position $i$, the measured signal consists of a direct propagation term and first-order (single-scattered) contributions, as illustrated in Fig.~\ref{fig:Diag}:
\begin{equation}
\vec m_i = \vec{g}_{ij}\ast\vec s_j + \sum_{k=1}^N \vec{g}_{ik} \ast \vec{d}_k \ast\vec{g}_{kj}\ast \vec s_j.
\end{equation}
Here, ``single scattering'' means first order in \emph{defect-induced} scattering. Boundary reflections and other multipath propagation of the undamaged plate are already embedded in the empirically measured impulse responses $\vec g_{ij}$; only terms with multiple defect interactions are neglected.

It can be rewritten in matrix form:
\begin{equation}
\vec m_i = G_{ij}\vec s_j + \sum_{k=1}^N G_{ik} D_k G_{kj} \vec s_j.
\end{equation}

Using the commutativity of circular convolution, restate the problem as
\begin{equation}
\Delta \vec m_{ij} := \vec m_i - G_{ij}\vec s_j = \sum_{k=1}^N (G_{ik} G_{kj} S_j)\, \vec{d}_k.
\end{equation}

Writing the defect vectors $\vec{d}_k$ as a single vector:
\begin{equation}
\vec{\mathbf{d}}_{\mathrm{full}} := \begin{bmatrix} \vec{d}_1 \\ \vec{d}_2 \\ \vdots \\ \vec{d}_N \end{bmatrix}
\end{equation}

The problem can be restated as a general matrix problem with

\begin{equation}
L^{\mathrm{gen}}_{ij} := \big[\; G_{i1} G_{1j} S_j \;\big|\; G_{i2} G_{2j} S_j \;\big|\; \cdots \;\big|\; G_{iN} G_{Nj} S_j \big]
\end{equation}
so that
\begin{equation}
\Delta \vec m_{ij} = L^{\mathrm{gen}}_{ij}\, \vec{\mathbf{d}}_{\mathrm{full}}.
\end{equation}
This system is typically highly underdetermined, and even when it is not, it is too large for practical computation. In our use case, the fully time‑resolved defect vector would have dimension $N_{\text{grid}} N_t \approx 10^6$ (with $N_{\text{grid}}$ spatial grid points and $N_t$ time samples per period).

However, from the assumption that scatterers generally re-emit waves almost immediately, we can restrict the defect vectors $\vec d_k$ around a small set of temporal basis functions. As mentioned in the previous section, we continue with the simplest case where $\vec{d}_k = d_k * \vec{e}_0$, i.e., a 1D basis. Since every defect at any location is parameterized by a single parameter $d_k$, the full defect vector can be represented as

\begin{equation}
\vec{\mathbf{d}} := \begin{bmatrix} d_1 \\ d_2 \\ \vdots \\ d_N \end{bmatrix}
\end{equation}
and integrated into the problem by including the basis vectors into the problem:


\begin{equation}\label{L_Matrix_short}
L_{ij} := \big[\; G_{i1} G_{1j} S_j \vec e_0 \;\big|\; G_{i2} G_{2j} S_j \vec e_0 \;\big|\; \cdots \;\big|\; G_{iN} G_{Nj} S_j \vec e_0 \big],
\end{equation}
where $\vec{e}_0$ in this equation is to be understood as a column vector shaped matrix. 
The final problem statement for one emitter at $i$ and sensor at $j$ is then
\begin{equation}\Delta \vec m_{ij} = L_{ij} \vec{\mathbf{d}}.
\end{equation}\label{partial_system}
For multiple sensor-emitter pairs, "stacking "the selected $(i,j)$ pairs yields the global residual system. For example for one emitter at (1) and two sensors at (2,3):
\begin{equation}\label{full_system}
\mathbf{ \vec m}_{\mathrm{full}} =
\begin{bmatrix}
\Delta \vec m_{1 2} \\
\Delta \vec m_{1 3} \\
\end{bmatrix}\qquad
L_{\mathrm{full}} =
\begin{bmatrix}
L_{1 2} \\
L_{1 3} \\
\end{bmatrix}
\qquad
\mathbf{\vec m}_{\mathrm{full}} = L_{\mathrm{full}} \vec{\mathbf{d}}
\end{equation}

Which corresponds to a [[1,2][1,3]] system. The general case can be much much larger, as in our cases presented, will be using 5 transducers with 10 total emitter-sensor pairings. (See appendix \ref{Appendix_sensor_configs} for details)
\paragraph{}

We have reduced the problem statement to the linear system in equation \eqref{full_system}. The only approximations made are the discretization of space, the assumption of first-order scattering and immediate reflection property of defects.



\subsection{Regularized inversion}\label{Regularized inversion}
We solve for $\vec{\mathbf{d}}$ in equation \ref{full_system}.  Let $\mathcal{L} = L_{\mathrm{full}}$ and $\mathbf{m} = \mathbf{m}_{\mathrm{full}}$ from equation \ref{full_system}. We solve the linear inverse problem by finding $\mathbf{\vec d}$ to minimize following expression:
\begin{equation}\label{eq:objective}
\text{argmin}_{\vec{\textbf{d}}} \; \|\mathcal{L}\mathbf{\vec d} - \mathbf{\vec m}\|_2^2 + \lambda \| \Gamma \mathbf{\vec d} \|_2^2
\end{equation}
where $\Gamma$ specifies the penalty metric (a weighting / masking matrix) and $\lambda>0$ controls the trade-off between data fit and stabilization. The normal equations yield the closed form \cite{tikhonovintro} \cite{Matrix_formalism_tikhonov}
\begin{equation}\label{eq:tik_gamma}
\vec{\mathbf{d}} = (\mathcal{L}^T \mathcal{L} + \lambda \Gamma)^{-1} \mathcal{L}^T \mathbf{\vec m}.
\end{equation}
The solution can computed directly iteratively, or by first computing the matrix $(\mathcal{L}^T \mathcal{L} + \lambda \Gamma)^{-1} \mathcal{L}^T$ (more expensive), and saving it. Allowing a rapid computation of $\vec{\mathbf{d}}$ though a matrix vector multiplication of many new measurement sets $\mathbf{\vec m}$.

If $\Gamma$ is the identity matrix, we recover standard Tikhonov regularization, which penalizes the Euclidean norm $\|\vec{\mathbf{d}}\|_2^2$. In practice, however, real systems can show small changes in pulse shape and amplitude between calibration and monitoring. The inversion would then explain these differences by placing strong ``defects'' directly at or near the transducer locations, which contaminates the rest of the reconstruction.

To isolate this effect, we choose $\Gamma$ to be diagonal. Entries corresponding to grid points at the transducers, and in a small neighborhood around them, are set to zero, all other diagonal entries are set to one. Coefficients in these unpenalized regions can adjust to absorb pulse variations without influencing the rest of the defect map. When visualizing the result, these regions are masked out (set to zero), since they mostly reflect pulse corrections rather than physical defects.

Sophisticated strategies for choosing the optimal regularization parameter $\lambda$ are discussed in \cite{Regularization_lambda_parameter_finding}, but for the present experimental and numerical examples they were not required. Here, $\lambda$ is selected by a simple trial‑and‑error sweep over a small set of candidate values, typically $\lambda \in \{10, 100, 1000\}$. Because all measured amplitudes are normalized as described in Section~\ref{Sec:Exp}, suitable values generally lie in this range.

\section{Experimental Setup}\label{Sec:Exp}
The test specimen is a $26.8\,\mathrm{cm} \times 26.8\ \times 0.14\,\mathrm{cm}$ bi-directional CFRP plate (8 layers, [0°\textsubscript{2}/90°\textsubscript{2}]\textsubscript{s}, HexPly 8552/AS4). Five low-cost brass–ceramic piezoelectric discs with a diameter of 12mm were bonded to the plate using a Crystalbond adhesive. As shown in Section \ref{Empirical impulse responses}, the method is independent of the source transfer function, and thus works for a wide range of unknown coupling properties.
\paragraph{}
The actuators were driven via a periodic pulse train with a $2\,\mathrm{ms}$ repetition period, using a 80\,kHz Ricker wavelet as the pulse shape. A 3D LDV was used to scan the plate on a $104\times104$ square grid, recording the full 3-component velocity vector over time. Figure~\ref{fig:Screenshot} shows one frame of the measurement. For this work, we use only the out of plane Vz component, which is excited strongest by the piezoelectric sources. The potential use of all three velocity components is discussed in \ref{Appendix_V_yxz}.
\paragraph{}
This scan was repeated for each of the five transducers, which constitute the one-time calibration step. Data were sampled at $625\,\mathrm{kHz}$ and low-pass filtered to 50\,kHz prior to computation. Each grid location was acquired with 10 repeats 5 separate times. I.e. we obtain 5 sets of measurements total, each having been averaged 10 times. All velocities are normalized by dividing all values by the maximum velocity recorded. The normalization increases numerical precision of floating point numbers in case of very high measurement numbers, and helps in having the optimal regularization parameter be of similar values for different systems, which will be introduced in section \ref{Regularized inversion}. As a result, the following plots such as Figure\ref{fig:Screenshot} use unitless velocities, and may not be consistent between different plots such as Figure\ref{fig:Dispersion:a} due to having a different global maxima chosen between different scripts. The normalization does not affect the results.

\begin{figure}
  \begin{subfigure}[t]{0.5663\linewidth}
    \centering
    \includegraphics[width=\linewidth]{Figures/IRL/Setup_Front.jpeg}
    \caption{}
    \label{fig:Setup:a}
  \end{subfigure}\hfill
  \begin{subfigure}[t]{0.4235\linewidth}
    \centering
    \includegraphics[width=\linewidth]{Figures/IRL/Setup_Back.jpeg}
    \caption{}
    \label{fig:Setup:b}
  \end{subfigure}
  
  \caption{
  Experimental setup for CFRP plate measurements.
  (a) Front view during LDV scanning; a white speckle coating is applied to increase surface reflectivity.
  (b) Rear view showing the five bonded transducers and the $1\,\mathrm{cm}^3$ metal cube used as a defect proxy; the central square mark is residue from prior tests.
  }
  \label{fig:Setup}
\end{figure}

\paragraph{}

After calibration, a $1\,\mathrm{cm}^3$ iron cube is bonded to the plate to simulate a defect. During monitoring, we excite each transducer again but measure only at the transducer locations. In our setup, the LDV acquires these sparse responses at the transducer positions. 
\paragraph{}
For future discussion, we roughly estimate the dispersion properties of the CFRP plate. We extract an x--t slice along the x--axis at y=6.2cm through the active transducer from Figure~\ref{fig:Screenshot} and window the time axis to exclude edge reflections (Figure~\ref{fig:Dispersion:b}). We then compute a 2D FFT to obtain the $k$--$\omega$ spectrum. Selecting the peak-energy wavenumber $k^*(f)$ for each frequency yields the phase velocity $v_p(f)=2\pi f/k^*(f)$ and wavelength $\lambda(f)=2\pi/k^*(f)$ (Figure~\ref{fig:Dispersion:a}). See Appendix~\ref{Appendix_Dispersion_Curve} for further comments regarding the validity of the phase velocities. 


\begin{figure}
  \begin{subfigure}[t]{0.48\linewidth}
    \centering
    \includegraphics[width=\linewidth]{Figures/Raw_data/150khz_Defect_0.17ms.png}
    \caption{Wavefield at t = 0.17ms without a defect yet present, but marked. This is considered the "calibration" scan that is fed into to the algorithm to learn the object properties.} 
    \label{fig:Screenshot}
  \end{subfigure}\hfill
  \begin{subfigure}[t]{0.48\linewidth}
    \centering
    \includegraphics[width=\linewidth]{Figures/Raw_data/defected_raw_data_0.21ms.png}
    \caption{Wavefield at t = 0.21ms, with the defect present. The algorithm does not have access to this scan, and serves for visual purposes.}
    \label{fig:Defected_Screenshot}
  \end{subfigure}
  
  \caption{Snapshots of the measured wavefield propagating in the CFRP plate, lowpass filtered to 150 kHz. Red diamonds mark the five transducers (numbered 1--5). Transducer 5 is the active source in this instance. The green square indicates the location where the defect was bonded in the monitoring phase (not yet present in this left calibration scan). Note: The results in this paper employ 1--50 kHz bandpass filtering, which significantly distorts the pulse shape compared to this view. (See Appendix~\ref{Appendix_50khz_pulse}.)
  }
  \label{fig:Raw_data}
\end{figure}

\begin{figure}
  \begin{subfigure}[t]{0.48\linewidth}
    \centering
    \includegraphics[width=\linewidth]{Figures/Dispersion_curve/space_time_slice_src5_xaxis.png}
    \caption{}
    \label{fig:Dispersion:b}
  \end{subfigure}\hfill
  \begin{subfigure}[t]{0.48\linewidth}
    \centering
    \includegraphics[width=\linewidth]{Figures/Dispersion_curve/dispersion_dual_axis_src5_xaxis.png}
    \caption{}
    \label{fig:Dispersion:a}
  \end{subfigure}
  
  \caption{
  Empirical dispersion of the CFRP plate obtained from the calibration scan.
  (a) Space–time diagram along the $y = 6.2\,\mathrm{cm}$ line through the active transducer (no.~5), derived from the dataset shown in Figure~\ref{fig:Screenshot}.
  (b) Phase velocity $v_p$ (blue, left axis, m/s) and wavelength $\lambda$ (red, right axis, mm) as functions of frequency $f$ (kHz); values are approximate due to material anisotropy.
  }
  \label{fig:Dispersion}
\end{figure}


\section{Results and Discussion}\label{Results}

This section presents the validation and evaluation of the proposed empirical operator framework through a sequence of numerical and experimental case studies. Rather than separating observations and interpretation, each subsection examines a specific regime relevant to non-destructive testing (NDT) and structural health monitoring (SHM), highlighting both performance and practical limitations of the method under controlled and realistic conditions.

\subsection{Baseline validation in a controlled numerical setting}\label{Simulation_Results}

We first assess the method in a controlled numerical environment to establish baseline performance under conditions consistent with the modeling assumptions. Elastic wave propagation is simulated using the spectral element modelling suite Salvus~\cite{mondaic}, considering a 50\,cm $\times$ 50\,cm aluminum plate with a thickness of 9\,mm. A localized tapered inclusion of 1.5\,cm diameter with relative density 0.2 is introduced to represent a small defect.

The excitation consists of a periodic pulse train with a 2\,ms repetition period using a 50\,kHz Ricker wavelet. Five transducers are arranged in a configuration comparable to the experimental setup, and simulated measurements include 20\% Gaussian RMS noise. The data are low-pass filtered to 0--50\,kHz to match the bandwidth used in subsequent experimental evaluations.

Figure~\ref{fig:Simulation_A_results} shows the reconstructed defect amplitude alongside the ground-truth density distribution. The defect is clearly localized at the correct position, demonstrating that the empirical operator formulation successfully recovers small scatterers when the assumptions of linearity, time invariance, and single scattering are satisfied. Background artifacts remain limited, indicating stable inversion behavior under controlled conditions.

This baseline result establishes a reference case against which deviations observed in experimental data and more challenging numerical scenarios can be interpreted.

\begin{figure}
  \begin{subfigure}[t]{0.48\linewidth}
    \centering
    \includegraphics[width=\linewidth]{Simulated_Figures/50Khz_2ms_defect_A.png}
    \caption{}   % <-- keeps (a)
    \label{fig:Simulation_A_results}
  \end{subfigure}\hfill
  \begin{subfigure}[t]{0.507\linewidth}
    \centering
    \raisebox{3.7pt}{\includegraphics[width=\linewidth]{Simulated_Figures/Defects/defect_A.png}}
    \caption{}   % <-- keeps (b)
    \label{fig:Simulation_A_Defect}
  \end{subfigure}
  \caption{
  Simulation case A with a single small inclusion.
  (a) Reconstructed defect amplitude in the 0--50\,kHz band obtained using the empirical-operator inversion, demonstrating accurate localization of a 1.5\,cm inclusion under 20\% RMS noise.
  (b) Ground-truth density map showing the tapered circular lower-density inclusion (relative density 0.2).
  }
  \label{fig:Defect_A}
\end{figure}


\subsection{Experimental defect localization with sparse measurements}\label{Results_Experimental}

We next evaluate the method experimentally using the CFRP plate described in Section~\ref{Sec:Exp}, thereby testing the calibration-to-monitoring workflow under real conditions. Following a one-time LDV calibration scan, defect localization is performed using only sparse measurements acquired at five transducer locations.

The experimental parameters closely follow the baseline simulation, with a 2\,ms repetition period and excitation by an 80\,kHz Ricker wavelet. Due to measurement noise at very low frequencies, the particle-velocity data are bandpass filtered to 1--50\,kHz. An iron cube ($1\times1\times1\,\mathrm{cm}^3$) bonded to the plate serves as a defect proxy.

Figure~\ref{fig:Results_normal} shows a clearly localized response at the defect position (yellow square). Cross-sectional profiles through the reconstruction along the x-axis are presented in Figure~\ref{fig:Results_side}. Computing the envelope slices exceeding 0.6 times the maximum amplitude yields an estimated localization half-width of approximately 2\,cm. Relative to a wavelength of $\lambda_{50}\approx8$\,cm at 50\,kHz (Figure~\ref{fig:Dispersion}), this corresponds to sub-wavelength localization. The physical interpretation of this behavior is discussed further in Appendix~\ref{Appendix_Sub-wavelentgh}.

Small black regions surrounding the transducers (red diamonds) are visible in Figure~\ref{fig:Results_normal}. These correspond to exclusion zones defined by the unpenalized region in the regularization (Section~\ref{Regularized inversion}) and are masked in the visualization. Without masking, these regions exhibit elevated amplitudes primarily caused by systematic experimental differences, such as pulse-amplitude variations between calibration and monitoring, rather than physical defect scattering.

The experimental result demonstrates that, once calibrated, the framework enables reliable defect localization using sparse monitoring measurements only. This separation between a data-intensive calibration stage and a lightweight monitoring stage is central to the intended SHM workflow and motivates further investigation of the method’s behavior across different frequency regimes and modeling assumptions.



\begin{figure}
  \begin{subfigure}[t]{0.48\linewidth}
    \centering
    \includegraphics[width=\linewidth]{Figures/Results/1-50khz + Defect.png}
    \caption{}
    \label{fig:Results_normal}
  \end{subfigure}\hfill
  \begin{subfigure}[t]{0.48\linewidth}
    \centering
    \raisebox{8pt}{\includegraphics[width=\linewidth]{Figures/Results/section_scan_1-50khz_x.png}}
    \caption{}
    \label{fig:Results_side}
  \end{subfigure}
  \caption{
  Defect localization results in the 1--50\,kHz band using sparse transducer measurements.
  (a) Reconstructed magnitude of the defect vector $|\vec{\mathbf{d}}|$ for the experimental CFRP plate, showing clear localization of the adhered metal cube near its true position (yellow square). Red diamonds indicate the five sensor locations; black regions surrounding them correspond to unpenalized exclusion zones introduced by the regularization.
  (b) Cross-sectional profiles along the x-axis through the reconstruction. Colored envelopes denote slices exceeding 0.6 times the maximum amplitude, yielding an estimated localization half-width of approximately $2\,\mathrm{cm}$.
  }
  \label{fig:Results}
\end{figure}



\subsection{Localization at very low frequencies (experimental limit)}\label{Results_LowFrequency}

To investigate the low-frequency limit of the method, we evaluate the experimental dataset in the 1--15\,kHz band, corresponding to the audible frequency range. In this regime, wavelengths approach the characteristic dimensions of the plate, providing a stringent test of localization capability under reduced information content.

Applying the inversion on the full $104\times104$ reconstruction grid yields mixed results (Appendix~\ref{Appendix_15khz}), characterized by unstable reconstructions and isolated high-amplitude pixels. To improve conditioning, the inversion is therefore restricted to a $50\times50$ spatial subgrid while using the same measurement data. The resulting reconstruction, shown in Figure~\ref{fig:15khz_subsampled}, exhibits a compact defect signature with an estimated width at half maximum of approximately 1.4\,cm, despite a wavelength on the order of $\approx16$\,cm in this frequency band.

This behavior highlights a practical conditioning effect: when the number of free defect parameters becomes large relative to the effective information content of the data, reconstructions become unstable and may develop localized artifacts. Reducing the number of spatial degrees of freedom improves stability by better matching the inversion space to the available temporal and spatial constraints, which are governed by bandwidth, repetition period, and sensor configuration. The result indicates that meaningful localization remains possible even at very low frequencies, provided that model complexity is appropriately balanced with data content.


\begin{figure}
  \begin{subfigure}[t]{0.48\linewidth}
    \centering
    \includegraphics[width=\linewidth]{Figures/Results/15khz_sub_no_def.png}
    \caption{}
    \label{fig:result_15khz_subsampled}
  \end{subfigure}\hfill
  \begin{subfigure}[t]{0.48\linewidth}
    \centering
    \raisebox{5pt}{\includegraphics[width=\linewidth]{Figures/Results/15khz_side.png}}
    \caption{}
    \label{fig:sideways_15khz_subsampled}
  \end{subfigure}
  \caption{
  Defect localization results in the 1--15\,kHz band using sparse transducer measurements and a $50\times50$ subsampled reconstruction grid.
  (a) Reconstructed magnitude of the defect vector $|\vec{\mathbf{d}}|$ for the experimental CFRP plate. Red diamonds indicate the five sensor locations; the adhered metal-cube defect is clearly resolved (marker omitted for visual clarity; see Figure~\ref{fig:Results_normal}).
  (b) Cross-sectional profiles along the x-axis through the reconstruction. Colored envelopes denote slices exceeding 0.5 times the maximum amplitude, yielding an estimated localization half-width of approximately $1.4\,\mathrm{cm}$.
  }
  \label{fig:15khz_subsampled}
\end{figure}

\subsection{Multiple-defect retrieval and limits of single scattering}\label{Results_MultipleDefects}

We next examine a scenario that departs from the single-defect baseline by introducing two tapered inclusions with diameters of 1.5\,cm and 5\,cm at relative density 0.2 (Figure~\ref{fig:Defect_A5B}). All simulation parameters are identical to those used in the baseline configuration.

The reconstruction successfully localizes both defects, demonstrating that the method can distinguish multiple scatterers and recover larger structures despite internally relying on a single-scattering formulation. However, pronounced background artifacts are observed compared to the single-defect case. These artifacts indicate degradation when the physical scenario deviates from conditions well described by the single-scattering approximation.

This case therefore illustrates a practical limitation of the current formulation: while localization remains feasible, reconstruction quality decreases as the scattering complexity increases.

\subsection{Resolution dependence on temporal bandwidth}\label{Results_Bandwidth}

To investigate whether additional temporal information mitigates the degradation observed above, the same two-defect configuration is simulated with an increased repetition period of 5\,ms and an extended frequency range of 0--100\,kHz. Relative to the baseline configuration, the algorithm therefore has access to approximately five times more temporal information (2\,$\times$ higher frequency bandwidth and 2.5\,$\times$ longer period).

The reconstruction shown in Figure~\ref{fig:Simulation_A5B_100khz_results} exhibits improved fidelity and substantially reduced background artifacts compared to the previous case. The longer listening window allows additional scattered energy to be captured within a single period, while the increased bandwidth enhances temporal resolution, together improving the separability of scattering contributions from multiple defects.

These observations indicate that part of the degradation seen in the multi-defect scenario arises from limited temporal information rather than solely from violations of the single-scattering assumption. Increasing bandwidth and observation time therefore sharpens localization and suppresses artifacts, representing a near-idealized scenario in which information limitations are reduced. A systematic analysis of these dependencies is deferred to~\cite{WueestMScThesis}.

\subsection{Single-transducer (pulse--echo) feasibility}\label{Results_ECHO}

Finally, we investigate a minimal sensing configuration in which a single transducer simultaneously acts as emitter and receiver, corresponding to a pulse--echo measurement geometry. This configuration represents an extreme reduction in spatial sampling and tests whether localization remains possible with strongly limited measurement aperture. This problem setting is related to those studied in \cite{ebrahimkhanlou_acoustic_2017,One_Channel_reversal}, enabling comparison with alternative approaches. 

The simulation setup follows the baseline validation case section \ref{Simulation_Results}, using a 5\,ms repetition period and a 0--100\,kHz bandpass filter with 10\% RMS noise, while all other parameters remain unchanged. The resulting reconstruction is shown in Figure~\ref{fig:ECHO}.

Despite the severe reduction in measurement information, the defect remains clearly localized. The reconstructed signature more closely resembles the high-bandwidth multi-defect result in Figure~\ref{fig:Simulation_A5B_100khz_results} than the baseline low-bandwidth case (Figure~\ref{fig:Simulation_A_results}), suggesting that localization performance is primarily governed by available frequency content rather than sensor count alone. However, more complex defect configurations reconstruct poorly in this pulse--echo geometry, and the solution becomes more sensitive to noise and degrades substantially at lower frequencies.

These observations indicate that while sparse or even single-sensor implementations are feasible in principle, their robustness depends strongly on bandwidth and signal quality. The behavior is consistent with findings reported in~\cite{Touchscreen}, where increased sensor counts mainly reduce background artifacts rather than fundamentally improving spatial resolution. Attempts to apply this configuration to real-world measurements did not yield interpretable results, consistent with the simulated sensitivity under comparable low-frequency and short-period constraints.

\subsection{Practical implications and limitations for SHM}\label{Discussion_SHM}

Across the numerical and experimental cases presented above, the empirical operator framework demonstrates consistent defect localization under sparse sensing conditions, while also revealing practical limits relevant for structural health monitoring (SHM) applications. The calibration-to-monitoring workflow enables a separation between a data-intensive initialization stage and rapid subsequent evaluations, suggesting suitability for repeated inspections once the system has been calibrated.

The reconstructed amplitudes throughout this work should be interpreted as relative indicators of defect location rather than quantitative scattering strengths, since the regularization parameter $\lambda$ influences amplitude scaling. Future implementations could address this dependence by explicitly characterizing the amplitude bias introduced by regularization and by accounting for grid-density effects in the inversion.

A comparison between simulations and experiments indicates broadly consistent behavior under comparable bandwidth and sparsity conditions. The experimental configuration differs slightly by restricting the usable band to 1--50\,kHz due to low-frequency measurement noise and by employing an adhered metal cube as a defect proxy rather than material density changes. Nevertheless, the agreement with the baseline simulation supports the numerical setup as representative of the method’s behavior under controlled conditions.

Several limitations emerge from the combined case studies. Reconstruction quality degrades when scattering complexity increases or when the single-scattering approximation becomes less valid, as observed in the multi-defect simulations. Likewise, when the number of free spatial parameters vastly exceeds the effective information content of the data, reconstructions may become unstable, particularly at low frequencies. While small defects can be localized well below the wavelength, accurate recovery of defect shape is not expected when defect dimensions are significantly smaller than the wavelength~\cite{WueestMScThesis}.

Finally, the experimental defect differs from realistic CFRP damage such as delaminations. However, it serves as a controlled proxy for contact-based perturbations. Together with the low computational cost after calibration (Appendix~\ref{Appendix_Computation}), the results suggest that the framework may enable practical SHM implementations based on sparse sensing, and potentially low-cost wave-based interfaces such as those explored in~\cite{Touchscreen}.

% \paragraph{}
% Further details and additional simulation examples are documented in the Appendix of \cite{WueestMScThesis}.

\begin{figure}
  \begin{subfigure}[t]{0.48\linewidth}
    \centering
    \includegraphics[width=\linewidth]{Simulated_Figures/A5B_50khz_2ms_0.2.png}
    \caption{}
    \label{fig:Simulation_A5B_results}
  \end{subfigure}\hfill
  \begin{subfigure}[t]{0.507\linewidth}
    \centering
    \raisebox{3.7pt}{\includegraphics[width=\linewidth]{Simulated_Figures/Defects/defect_A5B.png}}
    \caption{}
    \label{fig:Simulation_A5B_Defect}
  \end{subfigure}
  \caption{
  Simulation case A5B with two inclusions.
  (a) Reconstructed defect amplitude obtained using the empirical-operator inversion, showing simultaneous localization of the 1.5\,cm and 5\,cm inclusions together with increased background artifacts.
  (b) Ground-truth density map containing two tapered circular lower-density inclusions (relative density 0.2).
  }
  \label{fig:Defect_A5B}
\end{figure}


\begin{figure}
  \begin{subfigure}{0.48\linewidth}
    \centering
    \includegraphics[width=\linewidth]{Simulated_Figures/A5B_100khz_5ms_0.2.png}
    \caption{}
    \label{fig:Simulation_A5B_100khz_results}
  \end{subfigure}\hfill
  \begin{subfigure}{0.48\linewidth}
    \centering
    \includegraphics[width=\linewidth]{Simulated_Figures/Echo_5ms_N0.1.png}
    \caption{}
    \label{fig:ECHO}
  \end{subfigure}
  \caption{
  Simulation results illustrating the influence of increased temporal bandwidth and reduced sensor count.
  (a) Reconstruction of case A5B using a 5\,ms repetition period and 0--100\,kHz bandwidth, showing reduced background artifacts and sharper localization compared to Figure~\ref{fig:Defect_A5B}.
  (b) Single-transducer pulse--echo reconstruction of the small defect from case A (Figure~\ref{fig:Defect_A}), demonstrating successful localization using a 0--100\,kHz bandwidth with 10\% RMS noise.
  }
  \label{fig:Simulation_A5B_100khz_ECHO}
\end{figure}

\section{Conclusion}\label{Conclusion}

This work presented and experimentally validated an empirical, fully vectorized linear-operator framework for inverse scattering motivated by structural health monitoring applications. Source–receiver impulse-response operators are constructed using bonded low-cost piezoelectric transducers and a one-time LDV calibration scan. By enforcing a discrete periodic time axis, wave propagation is represented as DFT-diagonalizable circular convolution, allowing defect localization to be formulated as a Tikhonov-regularized linear inverse problem.

Experimental validation on an anisotropic CFRP plate demonstrated defect localization from sparse measurements acquired only at transducer locations after calibration. An adhered $1\,\mathrm{cm}^3$ iron cube was localized with sub-wavelength resolution, yielding a characteristic width of about $2\,\mathrm{cm}$ at a wavelength of approximately $8\,\mathrm{cm}$. Localization remained observable even in the very-low-frequency 1--15\,kHz band, where wavelengths approach the plate dimensions, provided that inversion complexity was matched to the available information content. Complementary simulations reproduced these observations under controlled conditions and enabled evaluation of practical limits, including multiple-defect scenarios, bandwidth dependence, and pulse–echo operation.

Across the investigated cases, reconstruction quality was found to depend primarily on the effective temporal information content—determined by bandwidth and listening window—rather than sensor count alone. While increased temporal bandwidth improves stability and suppresses background artifacts, performance degrades when the single-scattering assumption becomes less valid or when the number of reconstruction parameters exceeds the information supported by the data. Additional practical limitations arise from the simplified delta-like defect representation and the reliance on LDV measurements during calibration, which may introduce sensitivity to experimental alignment and measurement conditions.

The results demonstrate that reliable defect localization can be achieved using sparse measurements once empirical operators have been established. Requiring only linear time-invariant system behavior and no explicit forward model, the framework offers a data-driven alternative for applications where maintaining accurate digital twins is impractical. The demonstrated calibration-to-monitoring workflow provides a pathway toward practical SHM implementations enabling rapid, low-cost monitoring and potentially extending to other wave-based sensing and interaction systems.

\printcredits %Mandatory in the journal

%% Loading bibliography style file
%\bibliographystyle{model1-num-names}
\bibliographystyle{cas-model2-names}

% Loading bibliography database
\bibliography{cas-refs}

%% APPENDIX %%
\newpage
\appendix
\section{Additional experimental details and preprocessing}
\label{Appendix_A}

This appendix summarizes experimental processing choices relevant for interpreting the results, including the selected measurement component, filtering effects introduced by the periodic-time formulation, and auxiliary low-frequency evaluations.

\subsection{Longitudinal velocity components}
\label{Appendix_VelocityComponents}

Only the out-of-plane LDV velocity component $V_z$ is used throughout this work. This component is excited most strongly by the bonded piezoelectric transducers and provides the highest signal-to-noise ratio in the experimental dataset. 

The proposed framework itself is not restricted to a specific wave mode or measurement component, as it operates on an empirically calibrated linear time-invariant input–output relation. In principle, any measured observable (e.g., $V_x$, $V_y$, $V_z$, or transducer voltage) could be used. However, in the present experiments the in-plane components ($V_x$, $V_y$) exhibited substantially lower SNR and contained numerous unreliable or missing grid points, resulting in non-interpretable reconstructions. For this reason, all experimental results are restricted to $V_z$.

\subsection{Effect of 1--50\,kHz filtering under periodic time}
\label{Appendix_50khz_pulse}

The results in the main text employ a 1--50\,kHz bandpass filter, whereas the calibration snapshot shown in Figure~\ref{fig:Screenshot} is effectively unfiltered at 150 kHz. Figure~\ref{fig:50khz}(a) shows the same acquisition time after filtering. The reduced bandwidth increases the effective pulse duration, causing the response to extend over a larger fraction of the periodic time window.

As a consequence, near the end of the 2\,ms period ($t\approx1.9$\,ms), the tail of the current pulse overlaps with the onset of the next repetition, as illustrated in Figure~\ref{fig:50khz}(b). This apparent wraparound is a direct consequence of the discrete periodic-time formulation used throughout, and is consistent with the periodic implementation of the experiments. The framework therefore remains applicable even when pulses are not temporally isolated within a single period.

\subsection{Very-low-frequency evaluation (1--15\,kHz and 1--30\,kHz)}
\label{Appendix_15khz}

To further probe the low-frequency regime, the 1--15\,kHz and 1--30\,kHz bands were evaluated using the original full-resolution $104\times104$ defect grid (10\,816 unknowns). From Figure~\ref{fig:Dispersion}, the corresponding minimum wavelengths are approximately 16\,cm and 10\,cm, respectively.

The reconstructions shown in Figure~\ref{fig:15khz} demonstrate increasing instability at these frequencies. Although the defect region remains broadly identifiable, the images exhibit pronounced background artifacts and isolated high-amplitude pixels, particularly on the right side of the plate. This behavior is consistent with an underdetermined inversion in which the number of free defect parameters exceeds the effective information content of the data. In contrast, restricting the inversion to a reduced $50\times50$ grid in the main text (Figure~\ref{fig:15khz_subsampled}) produces a stable and compact localization.


\begin{figure}
  \begin{subfigure}[t]{0.48\linewidth}
    \centering
    \includegraphics[width=\linewidth]{Figures/Raw_data/50khz_0.17ms.png}
    \caption{}
    \label{fig:Screenshot_50khz}
  \end{subfigure}\hfill
  \begin{subfigure}[t]{0.48\linewidth}
    \centering
    \includegraphics[width=\linewidth]{Figures/Raw_data/50khz_1.9ms.png}
    \caption{}
    \label{fig:Screenshot_50khz_1.9ms}
  \end{subfigure}
  \caption{
  Effect of 1--50\,kHz bandpass filtering on the transducer pulse shape within the periodic time framework.
  (a) CFRP plate response at $t = 0.17\,\mathrm{ms}$ with transducer~5 active. Compared to the unfiltered snapshot in Figure~\ref{fig:Screenshot}, the pulse appears broadened due to low-pass filtering.
  (b) Response at $t = 1.9\,\mathrm{ms}$, near the end of the 2\,ms period. The tail of the current repetition and the onset of the next repetition are simultaneously visible, illustrating the wraparound inherent to the periodic time formulation.
  }
  \label{fig:50khz}
\end{figure}

\section{Additional diagnostics and limitations}
\label{Appendix_B}

This appendix summarizes secondary observations that influence reconstruction quality but are not central to the main methodological development.

\subsection{High-frequency degradation in the experiment}
\label{Appendix_Frequencies}

In the experimental example of Section~\ref{Results_Experimental}, extending the usable bandwidth beyond approximately 50\,kHz provides only limited improvement, and reconstructions degrade for frequencies above 100\,kHz. This behavior contrasts with the numerical simulations, where increased bandwidth consistently improves reconstruction quality.

The discrepancy is attributed to increased sensitivity of higher-frequency components to experimental imperfections, including calibration–monitoring misalignment, phase errors, and attenuation effects that are not fully captured by the empirical operator calibration. As a result, higher-frequency data contribute less useful information for defect localization in the present experimental configuration, whereas lower-frequency bands remain comparatively robust.

\subsection{Interpretation of sub-wavelength localization}
\label{Appendix_Sub-wavelentgh}

Several results in the main text exhibit localization widths smaller than the order-of-magnitude wavelength estimated from Figure~\ref{fig:Dispersion}. This observation should be interpreted cautiously. The presented framework performs a global inverse reconstruction rather than a conventional imaging process, and the resulting spatial localization reflects the stability of the inverse solution rather than a diffraction-limited resolution criterion.

Localization therefore remains frequency dependent, as demonstrated by the bandwidth and low-frequency studies, but is not directly constrained by classical imaging limits. A detailed theoretical analysis of this behavior is beyond the scope of the present work and is left for future investigation.


\begin{figure}
  \begin{subfigure}{0.48\linewidth}
    \centering
    \includegraphics[width=\linewidth]{Figures/Results/15khz.png}
    \caption{}
    \label{fig:result_15khz_bad}
  \end{subfigure}\hfill
  \begin{subfigure}{0.48\linewidth}
    \centering
    \includegraphics[width=\linewidth]{Figures/Results/30khz.png}
    \caption{}
    \label{fig:sideways_15khz}
  \end{subfigure}
  \caption{
  Low-frequency defect reconstructions on the full $104\times104$ grid.
  (a) Reconstruction using a 1--15\,kHz bandpass filter, showing isolated high-amplitude pixels and pronounced background artifacts.
  (b) Reconstruction using a 1--30\,kHz bandpass filter on the same grid.
  Compared to the subsampled $50\times50$ result in Figure~\ref{fig:15khz_subsampled}, the full-grid inversions exhibit pixel-scale spikes consistent with an overparameterized defect space at very low frequencies.
  }
  \label{fig:15khz}
\end{figure}

\section{Implementation aspects and extensions}
\label{Appendix_C}

This appendix summarizes computational characteristics of the implementation and outlines possible extensions of the empirical operator framework beyond the configurations considered in the main text.

\subsection{Computational performance}
\label{Appendix_Computation}

In the implementation used in this work, the global system matrix $\mathcal{L}_{\mathrm{full}}$ from Eq.~(14) is formed explicitly and inverted in a Tikhonov sense via
\[
(\mathcal{L}^T \mathcal{L} + \lambda \Gamma)^{-1}\mathcal{L}^T .
\]
For a matrix of size $n \times m$, where $n$ denotes the number of measurements and $m$ the number of defect parameters, memory requirements scale as $\mathcal{O}(nm)$ and inversion cost as $\mathcal{O}(nm^2)$. Increasing the number of measurements is therefore comparatively inexpensive, whereas enlarging the defect space dominates computational cost.

For the experimental CFRP case ($\sim10^3$ measurements and $\sim10^4$ defect parameters), the one-time inversion requires approximately 17\,s on a standard laptop (Intel i7 CPU, 16\,GB RAM). Computation of empirical impulse responses requires about 4\,s and scales approximately linearly with grid size. Once calibration is completed, subsequent evaluations reduce to a single matrix–vector multiplication and require only 3–5\,ms per update.

This separation between a moderately expensive calibration stage and rapid evaluation enables efficient repeated monitoring once the empirical operators have been established.

\subsection{Sensor configurations}
\label{Appendix_Sensors}

The sensor configurations used in this work are described in Section~3.6. Measurement pairs were selected to avoid redundant reciprocal combinations (e.g., $[i,j]$ and $[j,i]$), which provide limited additional information under the reciprocity assumption. The presented configuration was found sufficient to obtain stable reconstructions while maintaining sparse sensing conditions.

\subsection{Toward fully transducer-based monitoring}
\label{Appendix_Transducers}

The present study relies on LDV measurements during calibration to obtain dense wavefield sampling. In principle, a fully transducer-based implementation could replace the LDV by combining emitting and sensing transducers during calibration, allowing empirical operators to be constructed directly from embedded sensors. Such an approach would extend the method toward practical structural health monitoring scenarios in which only permanently installed transducers are available.

\subsection{Generalization to arbitrary defect bases}
\label{Appendix_Basis}

The results in this paper employ a delta-like defect basis. The formulation can be generalized by introducing a set of empirical basis vectors describing different defect responses. Let $\mathbf{b}_1,\mathbf{b}_2$ denote two such basis vectors and define the projector matrix
\[
P := [\mathbf{b}_1 \ \mathbf{b}_2].
\]
Replacing $\vec{e}_0$ in Eq.~(12) yields the modified operator
\[
L_{ij} :=
\begin{bmatrix}
G_{i1}G_{1j}S_jP \\
G_{i2}G_{2j}S_jP \\
\vdots \\
G_{iN}G_{Nj}S_jP
\end{bmatrix},
\]
and expands the defect vector to
\[
\mathbf{d} =
\begin{bmatrix}
d_{1,1} \\
d_{1,2} \\
d_{2,1} \\
\vdots \\
d_{N,2}
\end{bmatrix},
\]
where $k$ indexes spatial location and the second index denotes the selected basis element.

This extension is not explored further here but may become important for representing distributed defects such as delaminations that deviate from delta-like scattering behavior.


\end{document}

