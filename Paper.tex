%% 
%% Copyright 2019-2024 Elsevier Ltd
%% 
%% Version 2.4
%% 
%% This file is part of the 'CAS Bundle'.
%% --------------------------------------
%% 
%% It may be distributed under the conditions of the LaTeX Project Public
%% License, either version 1.2 of this license or (at your option) any
%% later version.  The latest version of this license is in
%%    http://www.latex-project.org/lppl.txt
%% and version 1.2 or later is part of all distributions of LaTeX
%% version 1999/12/01 or later.
%% 
%% The list of all files belonging to the 'CAS Bundle' is
%% given in the file `manifest.txt'.
%% 
%% Template article for cas-dc documentclass for 
%% double column output.

%\documentclass[a4paper,fleqn,longmktitle]{cas-dc}
\documentclass[a4paper,fleqn]{cas-sc}

%\usepackage[authoryear,longnamesfirst]{natbib}
%\usepackage[authoryear]{natbib}
\usepackage[numbers]{natbib}
\usepackage{comment} % Added for comment environment
\usepackage{etoolbox} % Added for \csdef command
\usepackage{xspace} % Added to support \xspace used in \tsc
\usepackage{subcaption} % Added for subfigures


%%%Author definitions
\def\tsc#1{\csdef{#1}{\textsc{\lowercase{#1}}\xspace}}
\tsc{WGM}
\tsc{QE}
\tsc{EP}
\tsc{PMS}
\tsc{BEC}
\tsc{DE}

% Define tblwidth if not already defined
\providecommand{\tblwidth}{\textwidth}
%%%

\begin{document}
\let\WriteBookmarks\relax
\def\floatpagepagefraction{1}
\def\textpagefraction{.001}
\shorttitle{Vectorized Linear Operator Formulation for Inverse Scattering; Draft version}
\shortauthors{R. G. Wueest et al.}

\title [mode = title]{Periodic Empirical Operator Framework for Inverse Scattering and Structural Health Monitoring}
%\title [mode = title]{Fully Vectorized Linear Operator Formulation for Inverse Scattering Problems}                      


\author[1]{R. G. Wueest}[type=editor,
                        role=Researcher]
\cormark[1]
\fnmark[1]
\ead{richard.wuest@protonmail.com}

%\credit{Conceptualization, Methodology, Software, Formal analysis, Investigation, Visualization, Writing – original draft, Writing – review \& editing}

%\address[1]{, Street 129, 1043 NX Amsterdam, The Netherlands}
\affiliation[1]{organization={Department of Physics, ETH Zürich},
%               citysep={}, % Uncomment if no comma needed between city and postcode
                postcode={8092}, 
                state={Zürich},
                country={Switzerland}}

\author[2]{J. Mueller}[role=Advisor]
\credit{Writing - review \\\& editing}


\affiliation[2]{organization={Department of Earth Sciences, ETH Zürich},
                postcode={8092}, 
                state={Zürich},
           country={Switzerland}}

% Additional advisors from thesis (placeholders where details absent)
\author[2]{J. Aichele}[role=Advisor]
\credit{Writing - review \& editing}


\author[2]{H. R. Thomsen}[role=Advisor]
\fnmark[2]
\ead{thomsenh@ethz.ch}
\credit{Supervision, Validation, Writing - review \& editing}


\begin{abstract}
We present an empirical, fully vectorized linear-operator framework for first order inverse scattering and validate it experimentally on a bidirectional carbon-fiber-reinforced polymer (CFRP) plate. The framework uses bonded piezoelectric transducers to emit periodic pulses and a Laser Doppler Vibrometer (LDV) for a one-time calibration scan, constructing source–receiver impulse-response operators. By enforcing a discrete, periodic time behavior, wave propagation is modeled as a circular convolution, diagonalizable via the discrete Fourier transform (DFT). Defect localization is then formulated as a Tikhonov-regularized linear inverse problem.

After calibration, monitoring uses only sparse LDV measurements at the transducer locations. An iron cube glued to the plate serves as a defect proxy and is localized to a sub wavelength resolution. Computation cost is concentrated to the one-time calibration, whereas, subsequent evaluations require minimal data and time. If the sparse measurements are implemented using piezoelectric sensors, it could enable fully automated structural health monitoring. The empirical-operator concept may extend to alternative detection systems or unconventional sensing interfaces (e.g., touch-like surfaces). The approach is model-free and broadly applicable where forward modeling or digital-twin upkeep is impractical.
\end{abstract}

\begin{graphicalabstract}

\end{graphicalabstract}

\begin{comment}
\begin{graphicalabstract}
\includegraphics{thumbnails/cas_email.jpeg}
\end{graphicalabstract}
\end{comment}

\begin{highlights}
\item Presents a fully vectorized linear operator framework for inverse scattering in NDT.
\item Relies solely on the principles of linear time-invariant (LTI) systems, avoiding the need for theoretical assumptions about wave propagation.
\item Enforces periodic discrete time, enabling FFT-diagonalizable circular convolution operators which are solvable using the Tikhonov regularized inverse.
\item Localizes sub-wavelength sized defects in both simulated aluminum plates and real-world CFRP plates.
\item Post-calibration, only sparse sensor measurements are needed for rapid health monitoring.
\end{highlights}

\begin{keywords}
Inverse scattering \sep Ultrasonic NDT \sep Data driven modeling\sep Defect localization \sep Empirical impulse response \sep Tikhonov regularization \sep Structural health monitoring
\end{keywords}


\maketitle

\section{Introduction}\label{Introduction}
% Structural Health Monitoring (SHM) of engineered components aims to localize and characterize defects. The importance of SHM depends on the cost of failure, which is especially relevant in aviation, hence many methods of performing SHM on flat structures have been explored and documented\cite{Aviation_Book}. Among these methods, Piezoelectric sensors are promising from a commercial point of view due to their low cost. However, increased challenges are noted when monitoring Carbon-fiber-reinforced polymer (CFRP) components \cite{Aircraft_NDT}. Different methods for CFRP SHM are explored specifically in \cite{Current_Methods}\cite{Composite_pipe_ndt}.

% Sparse monitoring using Piezoelectric sensors (measuring at a few strategically placed sensors) of lamb waves \cite{Lamb_wave_transducer} would therefore be especially desirable, but it typically requires detailed numerical forward models that are difficult to build and maintain in composites \cite{Modelling_Challenges}\cite{OG_Lamb_wave}. CFRP plates, in particular, exhibit variability from layup and manufacturing that undermines predictive digital twins \cite{Manufacturing_Variability_Problems}\cite{CFRP_Defects} . The use of Laser Doppler Vibrometers (LDV) in data-driven approaches for composite structures has gained increasing attention as a means of circumventing these challenges \cite{LDV_SHM_CFRP}. 

Structural Health Monitoring (SHM) of engineered components seeks to localize and characterize defects before they compromise structural integrity. The relevance of SHM is closely linked to the cost of failure, which is particularly critical in aviation, and has motivated extensive research on monitoring approaches for relatively simple, flat structures \cite{Aviation_Book}. Among the available techniques, piezoelectric transducers are attractive from a commercial perspective because of their low cost and ease of integration. Nevertheless, significant difficulties arise when these concepts are transferred to carbon-fiber reinforced polymers (CFRPs) \cite{Aircraft_NDT}, and a range of dedicated strategies for CFRP inspection have therefore been investigated \cite{Current_Methods,Composite_pipe_ndt}. Interpreting guided-wave measurements in such materials is inherently complex: overlapping modes often demand dedicated processin—such as ridge-based time–frequency analysis, beamforming, or migration—to extract meaningful arrivals \cite{Raghavan2007}, while benign geometric features, multiple reflections, and mode conversions at joints and thickness transitions generate coherent clutter that can obscure damage signatures \cite{Cawley2002,Olisa2021}. These effects are amplified by the pronounced anisotropy, laminate heterogeneity, and viscoelastic damping of CFRPs, which lead to direction- and frequency-dependent dispersion and attenuation \cite{Calomfirescu2008,Ostachowicz2024}. Curvature, stiffeners, and adhesive interfaces introduce additional localized scattering, and in large curved assets such as rocket fuselages or wind-turbine blades, spatially varying thickness and fiber orientation challenge travel-time–based concepts \cite{Yu2025,Chia2023,Jankauskas2011,Mustapha2024,Park2017}.

Sparse monitoring of Lamb waves using a limited number of piezoelectric sensors \cite{Lamb_wave_transducer} would therefore be highly attractive. However, such approaches generally rely on detailed numerical forward models. Modern numerical frameworks—including finite- and spectral-element solvers combined with higher-order plate formulations have proven capable of reproducing dispersion, mode conversion, and multi-path scattering in anisotropic laminates with high fidelity while remaining computationally efficient at ultrasonic frequencies \cite{Calomfirescu2008,Kudela2024}. Nevertheless, such models are difficult to maintain for composite structures \cite{Modelling_Challenges,OG_Lamb_wave}. CFRP components, in particular, exhibit variability arising from lay-up and manufacturing processes that undermines the reliability of predictive digital twins \cite{Manufacturing_Variability_Problems,CFRP_Defects}. In response to these limitations, data-driven strategies based on Laser Doppler Vibrometry (LDV) have gained increasing attention as a means to capture rich wavefield information directly from the structure and to reduce dependence on uncertain a priori models \cite{LDV_SHM_CFRP, Thomsen2024}.

This paper validates a novel method based on empirical linear operators\cite{Matrix_use_in_ndt}\cite{Touchscreen} and a fundamentally periodic time formulation. The result is a matrix based description of the system and it's inversion to solve for defects \cite{inversion_method_1}. We excite the structure with bonded piezoelectric transducers and use an LDV to perform a dense, one-time calibration LDV scan that constructs source–receiver impulse-response operators. Wave propagation is treated on a discrete cyclic time axis so that it acts as circular convolution, diagonalized by the discrete Fourier transform (DFT). 
By designing the excitation to be strictly periodic and ensuring no other sources act on the object, the cyclic time embedding is valid by construction. We impose time periodic behavior in the setup, and thus the frequency treatment becomes equivalent to the time domain treatment.
Within a single-scattering formulation, defect localization is solved via Tikhonov-regularized inverse\cite{Tikhonov}\cite{Tikhonov_matrix} using only sparse monitoring data at the transducer locations. We validate the approach experimentally on an anisotropic CFRP plate, demonstrating sub-wavelength localization from sparse measurements.


\section{Methodology}
The test specimen is a $26.8\,\mathrm{cm}\times26.8\ \times0.14\,\mathrm{cm}$ bi-directional CFRP plate (8 layers, [0°\textsubscript{2}/90°\textsubscript{2}]\textsubscript{s}, HexPly 8552/AS4). Five low-cost brass–ceramic piezoelectric discs with a nominal diameter of 12 mm are bonded to the plate. The drive is a periodic pulse train with a $2\,\mathrm{ms}$ repetition period, using a 80\,kHz Ricker wavelet as the pulse shape to enforce timing periodicity. Transducers were bonded using a hand-applied, temperature-activated adhesive (Crystalbond). Exact formulation and bond thickness were not recorded. As will be derived in Section \ref{Empirical impulse responses}, the method is independent of the source transfer function, and thus works for a wide range of unknown coupling properties. An LDV scans the plate on a $104\times104$ square grid, recording the full 3-component velocity vector over time. For this work, we use only the out of plane Vz component, which is excited strongest by the piezoelectric sources.   Figure~\ref{fig:Screenshot} shows one frame of such a measurement. The potential use of all three velocity components is discussed in \ref{Appendix_V_yxz}.


This scan is repeated for each of the five transducers, which constitute the one-time calibration step. Data are sampled at $625\,\mathrm{kHz}$ and low-pass filtered to 50\,kHz prior to computation. Each grid location is acquired with 10 repeats 5 separate times. I.e. we obtain 5 sets of measurements total, each having been averaged 10 times. All velocities are normalized by dividing all values by the maximum velocity recorded. The normalization increases numerical precision of floating point numbers in case of very high measurement numbers, and helps in having the optimal regularization parameter be of similar values for different systems, which will be introduced in section \ref{Regularized inversion}. As a result, the following plots such as Figure\ref{fig:Screenshot} use unitless velocities, and may not be consistent between different plots such as Figure\ref{fig:Dispersion:a} due to having a different global maxima chosen between different scripts. The normalization does not affect values of the results.


\begin{figure}
  \centering
  \includegraphics[width=0.5\linewidth]{Figures/Misc/LDV_Diag.png}
  \caption{Experimental setup: periodic excitation with a single active transducer; an LDV scans the plate to build empirical source--receiver operators during calibration. The source pulse is taken as the self-measured excitation (including local reverberations), and periodic timing is enforced by design.}
  \label{fig:Diag_LDV}
\end{figure}



After calibration, a $1\,\mathrm{cm}^3$ iron cube is bonded to the plate to simulate a defect. During monitoring, we excite each transducer again but measure only at the transducer locations. In our setup, the LDV acquires these sparse responses at the transducer positions. 

For future discussion regarding sub-wavelength localization, we estimate the dispersion properties of the CFRP plate. We extract an x--t slice along the x--axis at y=0.62m through the active transducer from Figure~\ref{fig:Screenshot} and window the time axis to exclude edge reflections (Figure~\ref{fig:Dispersion:b}). We then compute a 2D FFT to obtain the $k$--$\omega$ spectrum. Selecting the peak-energy wavenumber $k^*(f)$ for each frequency yields the phase velocity $v_p(f)=2\pi f/k^*(f)$ and wavelength $\lambda(f)=2\pi/k^*(f)$ (Figure~\ref{fig:Dispersion:a}). See Appendix~\ref{Appendix_Dispersion_Curve} for further comments regarding the validity of the phase velocities. 



\begin{figure}
  \centering
  \includegraphics[width=0.56\linewidth]{Figures/Raw_data/150khz_Defect_0.17ms.png}
  \caption{Snapshot of the CFRP plate at $t = 0.17$ ms, lowpass filtered to 150 kHz. Red diamonds mark the five transducers (numbered 1--5). Transducer 5 is the active source. The green square indicates the location where the defect was bonded in the monitoring phase (not yet present in this calibration scan). Note: The results in this paper employ 1--50 kHz bandpass filtering, which significantly distorts the pulse shape compared to this view. see Appendix~\ref{Appendix_50khz_pulse}.}
  \label{fig:Screenshot}
\end{figure}

\begin{figure}
  \begin{subfigure}{0.48\linewidth}
    \centering
    \includegraphics[width=\linewidth]{Figures/Dispersion_curve/space_time_slice_src5_xaxis.png}
    \caption{Space-time diagram along the y = 0.62m plane through the active sensor nr 5, derived from the same data as Figure \ref{fig:Screenshot}.}
    \label{fig:Dispersion:b}
  \end{subfigure}\hfill
  \centering
  \begin{subfigure}{0.48\linewidth}
    \centering
    \includegraphics[width=\linewidth]{Figures/Dispersion_curve/dispersion_dual_axis_src5_xaxis.png}
    \caption{Phase velocity $v_p$ (blue, left axis, m/s) and wavelength $\lambda$ (red, right axis, mm) versus frequency $f$ (kHz). }
    \label{fig:Dispersion:a}
  \end{subfigure}
  
  \caption{Empirical dispersion of the CFRP plate from the calibration scan.}
  \label{fig:Dispersion}
\end{figure}

\begin{figure}
  \begin{subfigure}{0.48\linewidth}
    \centering
    \includegraphics[width=\linewidth]{Figures/IRL/LDV_Frontal.jpeg}
    \caption{LDV setup from the front}
    \label{fig:LDV_Front}
  \end{subfigure}\hfill
  \centering
  \begin{subfigure}{0.48\linewidth}
    \centering
    \includegraphics[width=\linewidth]{Figures/IRL/LDV_Back_DOG.jpeg}
    \caption{LDV Setup from the back}
    \label{fig:LDV_Back}
  \end{subfigure}
  
  \caption{Ich bin mir nicht sicher, ob solche bilder auch enthalten sein sollen, oder eher im Appendix mit verweis? Ein bessered bild von der hinterseite der platte wäre evtl auch noch gut, obwohl ich richtig gerne den Hund drauf behalten würde:D}
  \label{fig:IRL}
\end{figure}


\section{Mathematical framework}\label{Mathematical framework}
\subsection{System assumptions}\label{System assumptions}
We assume: (i) linearity; (ii) time invariance; (iii) reciprocity; (iv) practical flatness (LDV access); and (v) the ability to control the excitation of the material relative to noise.

Assumptions (i), (ii), and (iii) are naturally satisfied for mechanical waves under standard conditions. Assumption (iv) refers to geometries where the object thickness is small relative to its lateral dimensions, typically valid for plate-like components such as aircraft fuselage panels and shell structures. Assumption (v) simply requires controlled excitation capability, which is standard in active NDT configurations using piezoelectric or electromagnetic transducers.

\subsection{Terminology and Notation}
We work with discrete-time vectors and matrices throughout. The spatial domain is discretized into a set of points indexed by integers without implying any particular geometric arrangement. For instance, a 100×100 scan grid of a square plate yields 10,000 points labeled 1 through 10,000, with the specific mapping arbitrary, but fixed.

When a transducer emits a signal at location $i$, represented by time-domain vector $\vec{s}_i$, the observed signal $\vec{m}_j$ at location $j$ is given by the convolution with the impulse response $\vec{g}_{ij}$ between those locations: $\vec{m}_j = \vec{g}_{ij} \ast \vec{s}_i$. This formulation separates spatial indices (subscripts) from the time-domain representation (vectors).

For computational implementation, these vectors correspond to discrete-time samples over a fixed acquisition window, with all operations respecting the imposed periodicity described in subsequent sections.


\subsection{Periodic time and discrete representation}\label{Periodic time and discrete representation}

Repeating the excitations enforces periodicity so that the time axis can be treated on a discrete cyclic domain. Any impulse response between locations $i$ and $j$ is then a circular-convolution operator $G_{ij}$, represented by a circulant (circular Toeplitz) matrix fully determined by its first column (the sampled impulse-response vector $\vec g_{ij}$). Explicitly, for $\vec g_{ij} = (g_0, g_1, \ldots, g_{n-1})^T$,
\begin{equation}
G_{ij} = \begin{pmatrix}
g_0 & g_{n-1} & \cdots & g_1 \\
g_1 & g_0 & \cdots & g_2 \\
\vdots & \vdots & \ddots & \vdots \\
g_{n-1} & g_{n-2} & \cdots & g_0
\end{pmatrix}
\end{equation}
so that $G_{ij}\vec v = \vec g_{ij} * \vec v$ (circular convolution). All such operators are diagonalized by the discrete Fourier transform (DFT), enabling component-wise multiplication in frequency.



\subsection{Empirical impulse responses}\label{Empirical impulse responses}
We do not separate an ``ideal'' injected waveform from local reverberations at a transducer. The self-measured periodic signal $\vec s_s$ is taken as the source vector. For a transducer $s$ and measurement point $i$ (grid or other transducer), the response is modeled as a circular convolution:
\begin{equation}\label{eq:empirical_conv}
\vec m_{i} = G_{is} \vec s_s
\end{equation}
where $G_{is}$ is circulant (first column $\vec g_{is}$). Directional (mode) decomposition is intentionally omitted because the LDV records only the scalar out-of-plane velocity at each sampled point.

To robustly estimate $\vec g_{is}$, we use repeated measurements. Let $\{(\vec s_{s,p}, \vec m_{i\mid s,p})\}_{p=1}^P$ denote $P$ repeated acquisitions. In the frequency domain (hats denote DFT components), for each frequency bin $f$ we solve the Tikhonov-regularized least-squares problem:
\begin{equation}\label{Tikhonov_impulse}
\min_{\hat{g}_{is}(f)} \sum_{p=1}^P \left| \hat{m}_{i\mid s,p}(f) - \hat{s}_{s,p}(f) \hat{g}_{is}(f) \right|^2 + \alpha |\hat{g}_{is}(f)|^2
\end{equation}
with regularization parameter $\alpha > 0$. The closed-form solution is:
\begin{equation}
\hat{g}^{\mathrm{reg}}_{is}(f) = \frac{\sum_p \overline{\hat{s}_{s,p}(f)}\, \hat{m}_{i\mid s,p}(f)}{\sum_p |\hat{s}_{s,p}(f)|^2 + \alpha}
\end{equation}
The result is a stabilized impulse-response function that does not rely on assuming specific noise characteristics.

\textbf{Notes:} Although the framework is presented with LDV-acquired empirical impulse responses, the same operators $G_{ij}$ can be generated computationally from an analytical model or a digital twin (e.g., FEM/FDTD). In that case, emitter and receiver coupling, as well as the enforced periodicity, must be accounted for separately. 


\subsection{Defect model}\label{Defect model}
Since defects also satisfy the assumptions in Section~\ref{System assumptions}, their effects can be modeled using circular convolutions. Within the single-scattering (Born-type) approximation, the scattering process is illustrated in Fig.~\ref{fig:Diag}. For clarity, we first assume a \emph{delta-like} (instantaneous) temporal reflection kernel, so each defect is represented by a scalar amplitude $d_k$ at its spatial location. Algebraically, this corresponds to selecting only the first canonical temporal basis vector $\vec e_0$. This is a \textbf{defect-basis choice} made for pedagogical clarity, not a limitation of the framework. The generalization to richer local bases is presented in Section~\ref{Generalization to arbitrary defect bases} after the main operator assembly and inversion.

\begin{figure}
  \centering
  \includegraphics[width=0.5\linewidth]{Figures/Misc/diagram.png}
  \caption{Visualization on how a source at location $j$ propagates to a defect at location $k$, which then re-emits the wave. Afterwards, it propagates to a receiver location $i$. As an equation, the image visualizes $\vec{m}_i = \vec{g}_{kj}\ast\vec{d}_k\ast\vec{g}_{ik}\ast\vec{s}_j$.}
  \label{fig:Diag}
\end{figure}



\subsection{Operator assembly}\label{Operator assembly}
For an emitter at $j$ and a measurement at $i$, the measured signal consists of a direct propagation term and first-order (single-scattered) contributions, as illustrated in Fig.~\ref{fig:Diag}:
\begin{equation}
\vec m_i = \vec{g}_{ij}\ast\vec s_j + \sum_{k=1}^N \vec{g}_{ik} \ast \vec{d}_k \ast\vec{g}_{kj}\ast \vec s_j.
\end{equation}
Which can be rewritten in matrix form:
\begin{equation}
\vec m_i = G_{ij}\vec s_j + \sum_{k=1}^N G_{ik} D_k G_{kj} \vec s_j.
\end{equation}

Using the commutativity of circular convolution, restate the problem as
\begin{equation}
\Delta \vec m_{ij} := \vec m_i - G_{ij}\vec s_j = \sum_{k=1}^N (G_{ik} G_{kj} S_j)\, \vec{d}_k.
\end{equation}

Writing the defect vectors $\vec{d}_k$ as a single vector:
\begin{equation}
\vec{\mathbf{d}}_{\mathrm{full}} := \begin{bmatrix} \vec{d}_1 \\ \vec{d}_2 \\ \vdots \\ \vec{d}_N \end{bmatrix}
\end{equation}

The problem can be restated as a matrix problem with

\begin{equation}
L^{\mathrm{gen}}_{ij} := \big[\; G_{i1} G_{1j} S_j \;\big|\; G_{i2} G_{2j} S_j \;\big|\; \cdots \;\big|\; G_{iN} G_{Nj} S_j \big]
\end{equation}
so that
\begin{equation}
\Delta \vec m_{ij} = L^{\mathrm{gen}}_{ij}\, \vec{\mathbf{d}}_{\mathrm{full}}.
\end{equation}
This system is typically highly underdetermined, and even when it is not, it is too large for practical computation. In our use case, the fully time‑resolved defect vector would have dimension $N_{\text{grid}} N_t \approx 10^6$ (with $N_{\text{grid}}$ spatial grid points and $N_t$ time samples per period).

However, knowing that scatterers generally re-emit waves almost immediately, we can restrict the defect vectors $\vec d_k$ around a small set of temporal basis functions. As mentioned in the previous section, we continue with the simplest case where $\vec{d}_k = d_k * \vec{e}_0$, i.e., a 1D basis. Since every defect at any location is parameterized by a single parameter $d_k$, the full defect vector can be represented as

\begin{equation}
\vec{\mathbf{d}} := \begin{bmatrix} d_1 \\ d_2 \\ \vdots \\ d_N \end{bmatrix}
\end{equation}
and integrated into the problem by including the basis vectors into the problem:


\begin{equation}\label{L_Matrix_short}
L_{ij} := \big[\; G_{i1} G_{1j} S_j \vec e_0 \;\big|\; G_{i2} G_{2j} S_j \vec e_0 \;\big|\; \cdots \;\big|\; G_{iN} G_{Nj} S_j \vec e_0 \big],
\end{equation}

where $\vec{e}_0$ in this equation is to be understood as a column vector. The final problem statement is then
\begin{equation}\Delta \vec m_{ij} = L_{ij} \vec{\mathbf{d}}.
\end{equation}

Stacking the selected $(i,j)$ pairs yields the global residual system
\begin{equation}\label{full_system}
\mathbf{m}_{\mathrm{full}} = L_{\mathrm{full}} \vec{\mathbf{d}},
\end{equation}
Here, "stacking" means vertically concatenating the pairwise residual vectors and their corresponding block-row operators, e.g.,
\[
\mathbf{m}_{\mathrm{full}} =
\begin{bmatrix}
\Delta \vec m_{1 2} \\
\Delta \vec m_{1 3} \\
\vdots
\end{bmatrix},\qquad
L_{\mathrm{full}} =
\begin{bmatrix}
L_{1 2} \\
L_{1 3} \\
\vdots
\end{bmatrix}
\]
How the stacking is performed in practice depends on how many sensors/emitters are available and how the system is set up.

We have reduced the problem statement to the linear system in \eqref{full_system}. The only approximations made are the discretization of space and the assumption of first-order scattering.


\subsection{Regularized inversion}\label{Regularized inversion}
Let $\mathcal{L} = L_{\mathrm{full}}$ and $\mathbf{m} = \mathbf{m}_{\mathrm{full}}$. We estimate the defect vector by minimizing the penalized least-squares functional. 
\begin{equation}\label{eq:objective}
J(\mathbf{d}) = \|\mathcal{L}\mathbf{d} - \mathbf{m}\|_2^2 + \lambda \| \Gamma^{1/2} \mathbf{d} \|_2^2,
\end{equation}
where $\Gamma \succeq 0$ specifies the penalty metric (a weighting / masking matrix) and $\lambda>0$ controls the trade-off between data fit and stabilization. The normal equations yield the closed form
\begin{equation}\label{eq:tik_gamma}
\hat{\mathbf{d}} = (\mathcal{L}^T \mathcal{L} + \lambda \Gamma)^{-1} \mathcal{L}^T \mathbf{m}.
\end{equation}
We use $\Gamma$ as a diagonal with ones everywhere except zeros at transducer locations along with a small exclusion zone around them, so those coefficients are not penalized. This allows the inversion to self-correct for slight inconsistencies in the emitter pulse form. When plotting the defect vector as a heatmap during evaluation, these regions are omitted from the plot by setting to 0, as otherwise, these regions would dominate the results.

The presented implementations of the Tikhonov regularized inversion is rather basic, avenues to be expanded upon \cite{Tikhonov_matrix}



\section{Results}\label{Results}
Figure~\ref{fig:Results_normal} shows the reconstructed defect vector $\vec{\mathbf{d}}$, where each pixel represents the absolute value at a spatial grid point. After solving equation~\ref{full_system}, the resulting vector assigns a reflection amplitude to every grid point. In this case, the vector has $101\times104 = 10,816$ entries, one per grid point. For visualization, each grid point is mapped back onto the 2D plane according to its original arrangement. The heatmap displays the absolute value at each location, which can be interpreted as the amplitude with which the incoming wave is reflected. No phase information is preserved, as the vector is complex-valued. The algorithm does not know the spatial correspondence of each index, and the system is not restricted to a 2D plane.

The defect in Figure~\ref{fig:Results_normal} is clearly localized near the target location (yellow square). Small black regions appear around each transducer (red diamonds). These exclusion zones correspond to the unpenalized region in the regularization (Section~\ref{Regularized inversion}) and are masked in the visualization. Behind this mask, the heatmap would show very high values, but these mostly reflect systematic errors in the experiment setup, such as differences in pulse amplitude between calibration and monitoring.

The result demonstrates successful localization of the $1\times1\times1\,\mathrm{cm}^3$ iron cube using only sparse measurements at the five transducer locations. The particle velocity data was bandpass filtered to 1--50 kHz and processed via the Tikhonov-regularized inversion described in Section~\ref{Regularized inversion}.

Figure~\ref{fig:Results_side} provides cross-sectional views along the x-axis. Slices exceeding 0.6 times the maximum amplitude are retained, and their envelope is drawn. This threshold identifies the localization width: the defect signature has a half-width of approximately 2 cm.

\begin{figure}
  \begin{subfigure}{0.48\linewidth}
    \centering
    \includegraphics[width=\linewidth]{Figures/Results/1-50khz + Defect.png}
    \caption{Reconstructed magnitude of the defect vector $|\vec{\mathbf{d}}|$ in the 1--50 kHz band. Red diamonds mark the five transducer locations. The yellow square indicates the actual location where the iron cube was bonded. The black regions around transducers are the unpenalized exclusion zones from the regularization.}
    \label{fig:Results_normal}
  \end{subfigure}\hfill
  \centering
  \begin{subfigure}{0.48\linewidth}
    \centering
    \includegraphics[width=\linewidth]{Figures/Results/section_scan_1-50khz_x.png}
    \caption{Cross-sectional profiles along the x-axis through the reconstructed defect (left panel). Colored envelopes show slices with amplitude exceeding 0.6 times the maximum. The localization width is estimated from the 0.6 amplitude threshold, yielding a half-width of approximately 2 cm.}
    \label{fig:Results_side}
  \end{subfigure}
  \caption{Defect localization results in the 1--50 kHz band using sparse transducer measurements. Left: full 2D reconstruction showing sub-wavelength localization of the defect. Right: cross-sectional analysis quantifying the localization width.}
  \label{fig:Results}
\end{figure}

\section{Discussion}\label{Discussion}

The reconstructed defect has a half-width of approximately 2 cm compared to a wavelength of $\lambda_{50} \approx 8$ cm at 50 kHz (Figure~\ref{fig:Dispersion:a}), yielding sub-wavelength localization. This is notably below the wavelength scale. The reasons for this property have prompted considerable discussion, and it remains challenging to provide a concise explanation, especially since it is difficult to demonstrate why something does \textit{not} occur. Based on the mathematical methodology used which does not rely on explicit spatial information, there appears to be no inherent restriction enforcing a wavelength limit. In contrast, other methods often have such restrictions, typically due to the need to reconstruct measured waves in space. For example, \cite{ebrahimkhanlou_acoustic_2017} describes a method with this limitation, though it is not explicitly discussed there.
\paragraph{}
However, in the simulation study, a decrease in resolution is observed at lower frequencies, which suggests that the method is not entirely independent of wavelength-related limits. The discussion is complicated by the fact that changing the frequency bandwidth simultaneously affects three factors: the maximum frequency, the amount of information available due to the system matrix shape (see Eq.~\ref{full_system}), and increased susceptibility to standard norm-based regularization, which can smear the result because neighboring points respond more similarly at longer wavelengths. For all intents and purposes, the method does have a resolution limitation as a function of frequency, but it is also dependent on other variables, without a hard cutoff based on frequency alone, such as with the Nyquist sampling theorem. Given the proof-of-concept nature of this paper, we refrain from exploring this topic in depth, simply noting that it is possible to isolate each variable to obtain definite answers if desired
\paragraph{}
The unpenalized region around transducers (Section~\ref{Regularized inversion}) was necessary to suppress phantom defects in early inversions. Pulse-shape measurements confirmed slight temporal and amplitude variations between calibration and monitoring phases. Rather than correcting these variations explicitly, the inversion absorbs them locally in the unpenalized zones.
\paragraph{}
Testing with extended frequency ranges showed that the 1--80 kHz band yields only modest quality improvements over 1--50 kHz, while extending to 1--150 kHz causes severe reconstruction degradation. Analysis of the 100--150 kHz band alone yields poor results, indicating that this frequency range carries no useful information for defect localization in the current setup. The 1--50 kHz range working similarly well to 1--80 kHz is especially interesting considering that the system matrix to be solved in equation~\ref{full_system} has dimensions $1000\times 10816$. The inversion problem is highly underdetermined by a factor of over 10. Further discussion can be found in Appendix~\ref{Appendix_Frequencies}.
\paragraph{}
A key source of degradation at higher frequencies is likely misalignment between the calibration and monitoring phases. During calibration, the LDV scans a dense 104×104 grid. During monitoring, the LDV measures at transducer locations only, and realignment relies on visual cues, introducing small spatial offsets. Such misalignments couple strongly to phase information at shorter wavelengths. Notably, this alignment issue would be eliminated entirely if monitoring were performed using the transducers themselves as receivers rather than the LDV, as discussed in Section~\ref{Appendix_Piezoelectric}.
\paragraph{}
The absolute pixel values in Figure~\ref{fig:Results_normal} should be interpreted as relative indicators of defect location rather than quantitative scattering amplitudes. The regularization parameter $\lambda$ significantly influences amplitude scaling, and the total defect effect depends on all spatial components jointly. Future quantitative work would require amplitude-density analysis and regularization-artifact correction. Here, the focus is on successful localization from sparse data.
\paragraph{}
Working effectively in the 1--50 kHz band validates the empirical-operator concept, even with practical imperfections such as pulse-shape drift and measurement misalignment. The robustness in this band supports our initial hypothesis: circumventing detailed forward models in favor of empirical operators leads to numerically robust properties in the detection algorithm, allowing imperfections to be absorbed into the interpretation of the measured quantities themselves. By removing explicit spatial information entirely (the algorithm does not know about the spatial relations between each grid point), it naturally removes restrictions caused by the frequency-to-pulse-width relation, allowing us to work at low frequencies, which in turn leads to numerical robustness.
\paragraph{}
Although not explicitly explored in this paper, previous  extensive simulation work documented in \cite{WueestMScThesis} indicated following limitations to the approach: Firstly, a large defect size relative ($\approx 0.1$) to the rest of the system becomes hard to resolve, likely caused by the single scattering approximation becoming less valid. Secondly, reconstruction quality depends on the defect shape.  Thirdly, while small defects can be localized well below wavelength sizes, the defect shape cannot be resolved if the defect is much smaller than the wavelength. 
\paragraph{}
In principle, the transducers themselves can act as receivers, which could enable fully automated, remote NDT/SHM. This could enable cost-effective, highly frequent SHM for important components at a press of a button. See Appendix~\ref{Appendix_Piezoelectric} for additional discussion of piezoelectric sensing considerations.

\section{Conclusion}\label{Conclusion}

This paper presented and experimentally validated an empirical, fully vectorized linear-operator framework for inverse scattering and structural health monitoring. Using bonded piezoelectric transducers and a one-time LDV calibration scan, we constructed source–receiver impulse-response operators on a discrete periodic time axis, represented propagation as circular convolution diagonalized by the DFT, and posed defect localization as a Tikhonov-regularized linear inverse problem. On an anisotropic CFRP plate, an adhered $1\,\mathrm{cm}^3$ iron cube was localized from sparse measurements at the transducer locations with sub-wavelength resolution ($\approx $2cm localization accuracy at a minimum wavelength of 8cm), concentrating computation in the one-time calibration while keeping subsequent monitoring lightweight.

The framework is general and modular: it does not rely on a detailed forward model and can, in principle, be extended to other defect bases, sensing layouts, and wave-bearing media, provided approximate linearity and time invariance. Current limitations, such as the single-scattering assumption, a simple defect basis, and reliance on LDV for calibration point to natural future work toward more complex scattering regimes and fully transducer-based implementations.

While the method presented here is intended as a practical route toward cost‑effective, frequent SHM, the broader point concerns how we choose to describe and interpret a physical system. Rather than insisting on a fully resolved forward model of every underlying quantity, we focus on the relationships between measured signals and let those relationships carry most of the modeling burden. We deliberately reshape the description: although physical time is not periodic, we can design an experiment that behaves as if it were, and then exploit the computational and experimental advantages this offers. In this description, many details that are usually modeled separately—transducer coupling, attenuation, and the interplay of multiple wave modes—are absorbed into the empirical operators by construction, or can be handled with minor adjustments. This perspective can extend beyond the linear‑operator framework used here. In SHM and NDT, periodic time can often be enforced experimentally, and the abstract treatment of measured quantities aligns naturally with recently popular machine‑learning approaches. Combining such experimental designs with data‑driven models may provide a general route to robust, low‑maintenance monitoring or detection systems in different fields.

 



\appendix
\section{Appendix}\label{My Appendix}

\section{Longitudinal Velocity Components}\label{Appendix_V_yxz}
In both simulation and real world experiments, only the out of plane components have been used. The real world setup used an LDV capable of also recording velocities in all three space directions. The method should in theory be able to use any of the three dimensions, or their sum as the measured quantities. This was not explored during the Simulation work, and in the real world experiment, these components showed high noise ratios, as well as many "dead" grid points without measurements for unknown systematic reasons, yielding non interpretable results. As such, we could not verify that this method extends to longitudinal wave components. 

Exploration of this property could be practical in the case of pursuit of a practical SHM implementation. If the hypothesis holds that the sum of all three velocity components is a valid quantity for this method, in the case of curved surfaces, would remove the need to compute the out-of-plane component at various directions. The result would be a simplified implementation where the three-component velocity at every grid point could be fed directly from the LDV into the algorithm without accounting for the orientation of the surface at any given point. Further reading: \cite{WueestMScThesis}.

\subsection{Low Frequency data}\label{Appendix_50khz_pulse}

The results presented in the main paper employ a 1--50 kHz bandpass filter, which differs significantly from the unfiltered calibration data shown in Figure~\ref{fig:Screenshot}. Figure~\ref{fig:Screenshot_50khz} shows the same acquisition time point, but after lowpass filtering to 50 kHz. The pulse duration expands dramatically, becoming so long that it wraps around within the periodic time window. At the end of the 2 ms period ($t \approx 1.9$ ms), the pulse from the next repetition is already visible (Figure~\ref{fig:Screenshot_50khz_1.9ms}). This behavior illustrates a key strength of the framework: the algorithm does not require short, well-separated pulses. The periodic time treatment is therefore not just a computational convenience but is fundamentally embedded in the mathematics of the inverse problem. 

\begin{figure}
  \begin{subfigure}{0.48\linewidth}
    \centering
    \includegraphics[width=\linewidth]{Figures/Raw_data/50khz_0.17ms.png}
    \caption{CFRP plate response at $t = 0.17$ ms, bandpass filtered to 1--50 kHz. Transducer 5 is active. Compared to the unfiltered snapshot in Figure~\ref{fig:Screenshot}, the pulse is significantly broadened due to the lowpass filtering.}
    \label{fig:Screenshot_50khz}
  \end{subfigure}\hfill
  \centering
  \begin{subfigure}{0.48\linewidth}
    \centering
    \includegraphics[width=\linewidth]{Figures/Raw_data/50khz_1.9ms.png}
    \caption{Same setup at $t = 1.9$ ms, near the end of the 2 ms period. The pulse tail from the current repetition is still visible, and the pulse from the next repetition (nominally starting at $t = 0.12$ ms) is already present, demonstrating the wraparound inherent in the periodic time treatment.}
    \label{fig:Screenshot_50khz_1.9ms}
  \end{subfigure}
  \caption{Effect of 1--50 kHz bandpass filtering on the transducer pulse shape in the periodic time framework.}
  \label{fig:50khz}
\end{figure}

\subsection{Results from simulations}\label{Appendix_Simulation_Results}
The method was validated using commercial finite-element simulations of a 9mm thick rectangular aluminum plate. Defects were modeled as localized reductions in density. 

\textbf{Case A}: Figure \ref{fig:Defect_A} contains a tapered circular lower-density inclusion of 1.5 cm diameter at 0.2 relative density relative to the surrounding. It demonstrates accurate localization in an controlled setting. The repetition period is 2ms, uses 5 sensors in a similar configuration, 20\% gaussian RMS noise,  and low pass filters the result to 0--50khz, similar to that of the real world example. It uses a 50khz Ricker wavelet as the driving pulse instead of 80khz, which should not significantly change results. 

\textbf{Case A5B\_1}: Figure \ref{fig:Defect_A5B} contains two tapered inclusions of 1.5 cm and 5 cm diameters at 0.2 relative density to the surrounding with the same parameters as before. It showcases the ability of the method to discern multiple defects simultaneously, as well as larger shapes, despite internally being built upon a single scattering assumption. The result does however show degradation of the reconstruction in the form of significant background artifacts, which appears to be a caused by the situation deviating from cases well described by single scattering.

\textbf{Case A5B\_2}: In Figure \ref{fig:Simulation_A5B_100khz_results}, the same defect as in \ref{fig:Defect_A5B} is presented again, but using 5ms periods with a 0--100khz frequency range. The Algorithm therefore has access to 5 times more information than all previous cases. We observe a significant increase in defect reconstruction fidelity, aswell as a significant decrease in background artifacts. Although the parameters are not isolated, we can compare the reconstruction of the small defect with that of Figure \ref{fig:Defect_A}, to see the increased localization accuracy as a result of higher frequency data available. The longer period, and therefore additional information available appears to help in compensating the multiple scattering effects, and thus reduces background artifacts. These claims are only partially derived from the results presented. The increase in period length to increase information is only possible up to the attenuation time of the waves, with the value of additional information decreasing as the remaining waves lose amplitude. The interpretation is derived from a high amount different simulation results using various parameters, which is discussed more in \cite{WueestMScThesis}. The results presented here alone do not justify these interpretations.

\textbf{Case ECHO}: Figure \ref{fig:ECHO} is a proof of concept that this method can work using only one sensor. In this case, only one sensor is active as emitter, and simultaneously acts as the only sensor, essentially a pulse-echo configuration. The data in this case is band passed to 100khz at 5ms period length, and only using 10\% RMS noise. Otherwise, the parameters are the same as the other simulation cases. We note that the reconstructed defect is very similar to the small defect of Figure \ref{fig:Simulation_A5B_100khz_results}, than that of \ref{fig:Simulation_A_results}, indicating that the localization accuracy is primarily determined by the data frequency, rather than the amound of sensors, and thus amount of data. However, while not shown here, more complex defects such as \ref{fig:Simulation_A5B_Defect} are evaluated much worse in this echo configuration. It is more susceptible to noise, and works significantly worse at lower frequencies. These findings appear to coincide with the description of a similar problem in \cite{Touchscreen}, that increased sensor counts decrease "contrast", called "background artifacts" in this paper, but do not increase resolution as higher frequencies do. The evaluation attempts using real world data did not yield interpretable results, likely due to aforementioned reasons.
\paragraph{}
This problem statement is very similar to that of \cite{ebrahimkhanlou_acoustic_2017}, providing a close comparison to a different method. Additionally, it addresses the single-channel time-reversal problem in a way that differs from conventional approaches \cite{One_Channel_reversal}. In many alternate formulations, the main difficulty lies in explicitly reversing the wavefield in time, whereas in the periodic-time framework used here, there is no intrinsic notion of “before’’ and “after’’. Time reversal is never posed as a separate operation, and the usual requirement of a time-reversal-invariant medium is removed. 
 

Further details and additional simulation examples are documented in \cite{WueestMScThesis}.

\begin{figure}
  \begin{subfigure}{0.48\linewidth}
    \centering
    \includegraphics[width=\linewidth]{Simulated_Figures/50Khz_2ms_defect_A.png}
    \caption{Reconstruction (simulated case A).}
    \label{fig:Simulation_A_results}
  \end{subfigure}\hfill
  \centering
  \begin{subfigure}{0.48\linewidth}
    \centering
    \includegraphics[width=\linewidth]{Simulated_Figures/Defects/defect_A.png}
    \caption{Ground-truth density map for case A: a tapered circular lower-density inclusion (1.5 cm diameter).}
    \label{fig:Simulation_A_Defect}
  \end{subfigure}
  \caption{Simulation case A. Left: reconstruction. Right: density map. An ideal case scenario similar to the real world example.}
  \label{fig:Defect_A}
\end{figure}

\begin{figure}
  \begin{subfigure}{0.48\linewidth}
    \centering
    \includegraphics[width=\linewidth]{Simulated_Figures/A5B_50khz_2ms_0.2.png}
    \caption{Reconstruction (simulated case A5B).}
    \label{fig:Simulation_A5B_results}
  \end{subfigure}\hfill
  \centering
  \begin{subfigure}{0.48\linewidth}
    \centering
    \includegraphics[width=\linewidth]{Simulated_Figures/Defects/defect_A5B.png}
    \caption{Ground-truth density map for case A5B: two tapered lower-density inclusions (1.5 cm and 5 cm diameters).}
    \label{fig:Simulation_A5B_Defect}
  \end{subfigure}
  \caption{Simulation case A5B. Left reconstruction. Right: Density map. }
  \label{fig:Defect_A5B}
\end{figure}


\begin{figure}
  \begin{subfigure}{0.48\linewidth}
    \centering
    \includegraphics[width=\linewidth]{Simulated_Figures/A5B_100khz_5ms_0.2.png}
    \caption{Reconstruction (simulated case A5B).}
    \label{fig:Simulation_A5B_100khz_results}
  \end{subfigure}\hfill
  \centering
  \begin{subfigure}{0.48\linewidth}
    \centering
    \includegraphics[width=\linewidth]{Simulated_Figures/Echo_5ms_N0.1.png}
    \caption{Defect A (Fig \ref{fig:Defect_A} reconstruction using only one transducer, using a 0--100kHz band.)}
    \label{fig:ECHO}
  \end{subfigure}
  \caption{TODO}
  \end{figure}



\subsection{Discussions regarding validity of the Dispersion curve}\label{Appendix_Dispersion_Curve}
Figures \ref{fig:Dispersion} suggest phase velocities that increase with frequency and reach values of order 10km/s at the upper end of the measured band. A theoretical Lamb-mode estimate\cite{Lamb_wave_Calculator} using rough CFRP parameters places the high-frequency A0 branch near ~3.8 km/s. Both the empirical and theoretical numbers should be considered approximate.

The empirical estimates may be affected by measurement noise, limited angular sampling, and small alignment errors between calibration and monitoring. Conversely, the theoretical Lamb-mode calculation is only a rough ballpark because it assumes isotropic, homogeneous material properties and does not account for the plate's anisotropy or layup. Hence discrepancies between the two are to be expected and can be substantial. Neither approach alone provides a definitive ``ground truth.''

For the purposes of this paper we only use phase-velocity/wavelength estimates as a rough reference to assess whether the observed localization is below the order‑of‑magnitude wavelength. At 50 kHz, the corresponding wavelengths are on the order of a few centimeters, and the observed localization ($\approx$2 cm) remains smaller than those wavelengths. Given that this comparison is only qualitative, we do not pursue a detailed velocity validation here.

\subsection{Regarding Frequencies and reconstruction}\label{Appendix_Frequencies}
The algorithm shows limited additional benefit from including higher-frequency data, even though the measured wave spectrum extends well beyond 50 kHz, as expected from the 80 kHz central frequency of the transmitted pulses and confirmed by measurements at multiple locations on the plate. This behavior deviates from previous simulation results obtained using the same method (Appendix~\ref{Appendix_Simulation_Results}, \cite{WueestMScThesis}).

As discussed in Section~\ref{Discussion}, sensor misalignment is likely to have a greater impact at higher frequencies. Additionally, the discrepancy may stem from assumptions in the problem formulation. In Section~\ref{Defect model}, defects are assumed to exhibit a delta-like response, implying instantaneous, frequency-independent reflection. In the simulation study, defects were modeled as changes in density, whereas the experimental defect, an adhered metal cube, more closely represents a change in stiffness, for which this assumption may be less valid.

In this context, restricting the frequency band may be advantageous: although fewer data are used, the underlying assumptions only need to hold over a narrower frequency range. However, the current experimental setup does not allow these effects to be isolated, so no definitive conclusion can be drawn.


\subsection{Piezoelectric Measuring}\label{Appendix_Piezoelectric}
%This section will likely be moved to a future work section, or into discussion.
***This section will likely have to be moved to a future work section or similar***
 A fully transducer-based implementation could proceed as follows: For each transducer location, two transducers are placed close together, one acting as the transmitter, and the other as the sensor. During calibration, a transmitting transducer emits a periodic pulse while an LDV scans the object. The measurements from the sensing transducer are added to the LDV data and treated as an additional grid point. In the code, the sensing transducer's measurement is assigned to the emitter location, so any coupling effects are absorbed into the object properties, potentially allowing the use of inexpensive transducers with poor coupling.

 	\textbf{Notes:} Thick, cube-shaped transducers should be avoided, as their stiffness locally alters the plate and makes measurement at or by the transducer impossible. Always use flat, plate-like ceramic transducers relative to the plate thickness.

 It is important to note that the physical location of the emitting transducer and the modeled transmitter location may not coincide. This is intentional and consistent with the approach in this paper, since the LDV does not scan exactly at the transducer positions. Whilal and modeled transducer locations is theoretically possible, it quickly becomes unstable at separations of a few centimeters. Using one emitter and multiple nearby sensing transducers, treating each sensor as an emitter in the model, may be feasible, but the practical benefit is likely limited.


\subsection{Generalization to arbitrary defect bases}\label{Generalization to arbitrary defect bases}
As mentioned in Section \ref{Defect model} and applied in Section \ref{Operator assembly}, we assumed the defect to act as a perfect reflector of form $\vec{e}_0$. In general this assumption does not hold for more complicated defects, such as delaminations in fiber-reinforced materials. In this case, using a simulation with a defect at a known location, one can compute the response nature through similar means to those described in Sections \ref{Operator assembly} and \ref{Regularized inversion}.
Assuming we have obtained empirical defect vectors for two defect types of interest, $\vec{b}_1$ and $\vec{b}_2$, we can implement these two vectors as a basis using a projector matrix $P$:
\begin{equation}
  P := \begin{bmatrix}
    \vec{b}_1 & \vec{b}_2
  \end{bmatrix}
\end{equation}
Replacing $\vec{e}_0$ in \ref{L_Matrix_short}, we obtain
\begin{equation}
L_{ij} := \big[\; G_{i1} G_{1j} S_j P \;\big|\; G_{i2} G_{2j} S_j  P \;\big|\; \cdots \;\big|\; G_{iN} G_{Nj} S_j  P \big],
\end{equation}
as well as an adjustment to the full defect vector $\mathbf{d}$ to hold two parameters per location.

\begin{equation}
\mathbf{d} := \begin{bmatrix} d_{1,1} \\ d_{1,2} \\ d_{2,1} \\\vdots \\ d_{N,2} \end{bmatrix}
\end{equation}
where for $d_{k,b}$, $k$ denotes the grid location, and $b$ the basis vector at that location.

This approach is not used in this paper, as we continue using a delta-like simplifies defect. Experiments in simulation work so far has not yielded many useful bases that significantly increased performance. Especially considering the computational costs associated with increasing the defect space, as well as since we are already working in highly underdetermined systems in this paper. See \cite{WueestMScThesis} for further discussions into different defect bases.

%\printcredits

%% Loading bibliography style file
%\bibliographystyle{model1-num-names}
\bibliographystyle{cas-model2-names}

% Loading bibliography database
\bibliography{cas-refs}


%\vskip3pt


\end{document}

